\documentclass[11pt,a4paper]{report}

\usepackage{float} %dzieki temu pakietowi rysunki pojawiaja sie tam gdzie chce po uzyciu opcji H w srodowisku figure
\usepackage[font=small,labelfont=bf]{caption}
\usepackage{natbib}
\usepackage{import}
\usepackage[tight]{subfigure}
\usepackage{tikz}
\usepackage{pgfplots}
\usepackage{polski}
\usepackage[cp1250]{inputenc}
\usepackage{indentfirst}
\usepackage{color}
\usepackage{graphics}
\usepackage{geometry}
\usepackage{amsmath,amssymb,amsfonts}
% ---- PAGE LAYOUT ----
\geometry{a4paper, top=1.5cm, bottom=1.5cm, left=1.5cm,
right=1.5cm, nohead}

\begin{document}

\begin{center}
\large{\textbf{ZESTAW ZADA� XIV}}
\end{center}

\noindent
\textbf{Zadanie 1} Oblicz ca�ki nieoznaczone:\\
\textbf{(a)} $\int \sin^4 x dx$, \textbf{(b)} $\int \sin^3 x \cos^2 x dx$, \textbf{(c)} $\int \sin^4 x \cos^2 x dx$, 
\textbf{(d)} $\int\frac{dx}{\sin x}$, \textbf{(e)} $\int \frac{(2\sin^2 x - 1)dx}{\sin x \cos x + \cos^2 + 1}$.\\

\noindent
\textbf{Zadanie 2} Oblicz ca�ki nieoznaczone:\\
\textbf{(a)} $\int\frac{dx}{\sqrt{-x^2+4 x+5}}$, 
\textbf{(b)} $\int\frac{dx}{\sqrt{x^2-8 x+25}}$, 
\textbf{(c)} $\int \frac{2 x^2+1}{\sqrt{x^2-8 x+25}} dx$,
\textbf{(d)} $\int \frac{3 x^3-10 x^2+4 x-5}{\sqrt{-x^2+4 x+5}} dx$.\\

\end{document}