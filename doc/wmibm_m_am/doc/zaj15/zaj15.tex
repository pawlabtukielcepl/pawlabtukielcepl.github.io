\documentclass[11pt,a4paper]{report}

\usepackage{float} %dzieki temu pakietowi rysunki pojawiaja sie tam gdzie chce po uzyciu opcji H w srodowisku figure
\usepackage[font=small,labelfont=bf]{caption}
\usepackage{natbib}
\usepackage{import}
\usepackage[tight]{subfigure}
\usepackage{tikz}
\usepackage{pgfplots}
\usepackage{polski}
\usepackage[cp1250]{inputenc}
\usepackage{indentfirst}
\usepackage{color}
\usepackage{graphics}
\usepackage{geometry}
\usepackage{amsmath,amssymb,amsfonts}
% ---- PAGE LAYOUT ----
\geometry{a4paper, top=1.5cm, bottom=1.5cm, left=1.5cm,
right=1.5cm, nohead}

\begin{document}

\begin{center}
\large{\textbf{ZESTAW ZADA� XV}}
\end{center}

\noindent
\textbf{Zadanie 1} Oblicz ca�ki oznaczone:\\
\textbf{(a)} $\int\limits_{1}^{2}\left(x^2 + \frac{1}{x^2}\right) dx$, 
\textbf{(b)} $\int\limits_{1}^{2} \frac{dx}{\sqrt{5x-1}}$ , 
\textbf{(c)} $\int\limits_{1}^{\sqrt{3}}x\,\text{arctg}\,x dx$,
\textbf{(d)} $\int\limits_{1}^{2} \frac{x dx}{x^2 - 9}$.\\


\noindent
\textbf{Zadanie 2} Wyznacz pola zawarte pomi�dzy liniami:\\
\textbf{(a)} $y = x^2$, $y = x + 2$, \textbf{(b)} $y = 4x - x^2$, $y = x^2 - 4x + 6$, \textbf{(c)} $y = \ln x$, $x + y = 1$, $y = 1$, \textbf{(d)} $y = \frac{1}{x^2}$, $y = 0$, $x\geq 1$.\\

\noindent
\textbf{Zadanie 3} \\
\textbf{(a)} oblicz obj�to�� bry�y powsta�ej przez obr�t krzywej $y = e^x$, wok� osi $Ox$, gdy $0\leq x\leq \ln 2$,\\
\textbf{(b)} oblicz obj�to�� bry�y powsta�ej przez obr�t elipsy $\frac{x^2}{a^2} + \frac{y^2}{b^2} = 1$, wok� osi $Ox$, $a>0, b>0$.\\


\noindent
\textbf{Zadanie 4} \\
\textbf{(a)} oblicz d�ugo�� �uku krzywej $y = \ln(\cos x)$, gdy $0\leq x \leq \frac{\pi}{3}$,\\
\textbf{(b)} wyznacz po�o�enie �rodka ci�ko�ci jednorodnego obszaru ograniczonego krzywymi $y = x^2$, $y = x+2$. \\

\end{document}