\documentclass[11pt,a4paper]{report}

\usepackage{float} %dzieki temu pakietowi rysunki pojawiaja sie tam gdzie chce po uzyciu opcji H w srodowisku figure
\usepackage[font=small,labelfont=bf]{caption}
\usepackage{natbib}
\usepackage{import}
\usepackage[tight]{subfigure}
\usepackage{tikz}
\usepackage{pgfplots}
\usepackage{polski}
\usepackage[cp1250]{inputenc}
\usepackage{indentfirst}
\usepackage{color}
\usepackage{graphics}
\usepackage{geometry}
\usepackage{amsmath,amssymb,amsfonts}
% ---- PAGE LAYOUT ----
\geometry{a4paper, top=1.5cm, bottom=1.5cm, left=1.5cm,
right=1.5cm, nohead}

\begin{document}

\begin{center}
\large{\textbf{ZESTAW ZADA� VI}}
\end{center}

\noindent
\textbf{Zadanie 1} Sprawd�, �e podane funkcje spe�niaj� podane r�wnania r�niczkowe:\\
\textbf{(a)} $y = e^{-3x}$, r�wnanie: $y' + 3y = 0$, \textbf{(b)} $y = 3\cos(5x) + 5\sin(5x)$, r�wnanie $y'' + 25y = 0$, \\
\textbf{(c)} $y = 3e^{-x} + 5xe^{-x}$, r�wnanie: $y''+2y'+y = 0$,\\ 
\textbf{(d)} $y = e^{-2x}(3\cos(3x) + 2\sin(3x))$, r�wnanie: $y'' + 4y' + 13y = 0$.\\

\noindent
\textbf{Zadanie 2}\\
\textbf{(a)} Zapisz wz�r Taylora dla funkcji $f(x) = \ln(x+1)$ w okolicy $x_0 = 0$ z dok�adno�ci� do $4$ wyraz�w; wykorzystaj otrzymany wz�r do obliczenia przybli�enia $\ln 1,1$ (warto�� wskazana przez kalkulator: $0{,}0953102$),\\
\textbf{(b)} Zapisz wz�r Taylora dla funkcji $f(x) = \frac{2x}{2-x}$ z dok�adno�ci� do dw�ch wyraz�w w okolicy $x_0 = 1$; wykorzystaj otrzymany wz�r do przybli�enia warto�ci funkcji dla $x = 0{,}9$,\\
\textbf{(c)} W oparciu o wz�r Taylora przybli� funkcj� $y = \sqrt{8 - x^2}$ w okolicy $x_0 = 2$ za pomoc� paraboli; sprawd� dok�adno�� przybli�enia w punktach $x = 2{,}5$ oraz $x = 2{,}1$,\\
\textbf{(d)} Zapisz wz�r Taylora dla funkcji $f(x) = \arcsin x$ z dok�adno�ci� do wyraz�w rz�du $3$--ego. Wykorzystaj otrzymany wz�r do obliczenia przybli�onej warto�ci $\arcsin(0{,}1)$ (warto�� wskazana przez kalkulator: $0{,}100167$).\\



\end{document}