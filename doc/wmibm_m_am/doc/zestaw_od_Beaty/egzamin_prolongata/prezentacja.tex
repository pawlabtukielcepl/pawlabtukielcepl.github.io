\documentclass{beamer}

\usetheme{Berlin}

\usepackage{polski}
\usepackage[cp1250]{inputenc}


\title{Egzamin z Analizy Matematycznej (sesja poprawkowa)}
\date{13/02/2023}

\begin{document}
	
\frame{\titlepage}

\begin{frame}[t]
\frametitle{Zadanie 1}

\noindent
\Large{
Oblicz pochodne: 
(a) $\left(\frac{1}{x^2} - \frac{1}{\sqrt[3]{x^2}}\right)'$,  
(b) $\left( \ln\frac{x^3 + 3x}{x^2 + 1} \right)'$.
}

\end{frame}

\begin{frame}[t]
\frametitle{Zadanie 2}

\noindent
\Large{
Oblicz ca�ki (a) $\int \frac{x dx}{\sqrt{x^2+4}}$, (b) $\int x \cos x dx$.
}

\end{frame}

\begin{frame}[t]
\frametitle{Zadanie 3}

\noindent
\Large{
Zapisz wielomian Taylora stopnia II--go dla funkcji $y = \ln x$ w okolicy $x_0 = 1$. Oblicz za jego pomoc� przybli�on� warto�� $\ln 0{,}9$.
}
\end{frame}

\begin{frame}[t]
\frametitle{Zadanie 4}

\noindent
\Large{
Wyznacz przedzia�y monotoniczno�ci i ekstrema lokalne funkcji: 
$y = x^4+2 x^3-4 x^2-12 x$.
}

\end{frame}

\begin{frame}[t]
\frametitle{Zadanie 5}

\noindent
\Large{
Wyznacz przybli�on� warto�� ca�ki $\int\limits_{0}^{4} (4x-x^2) dx$ dziel�c przedzia� ca�kowania na cztery r�wne cz�ci, za punkty po�rednie przyj�� �rodki kolejnych przedzia��w. Obliczenia prowadzi� na u�amkach zwyk�ych, wynik poda� w takiej samej postaci.
}


\end{frame}

\begin{frame}[t]
\frametitle{Zadanie 6}

\noindent
\Large{
Wyznacz pole obszaru ograniczonego liniami  $y = x^2 + 2x$, $y = 4x + 3$. Wykonaj rysunek!
}

\end{frame}

\begin{frame}[t]
\frametitle{Zadanie 7}\


\noindent
\Large{
Oblicz $z''_{xy}$, je�li $z = (x^2 + y^2 + 1)^2$.
}

\end{frame}


\end{document}