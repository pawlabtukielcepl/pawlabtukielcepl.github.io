\documentclass[11pt,a4paper]{report}

\usepackage{float} %dzieki temu pakietowi rysunki pojawiaja sie tam gdzie chce po uzyciu opcji H w srodowisku figure
\usepackage[font=small,labelfont=bf]{caption}
\usepackage{natbib}
\usepackage{import}
\usepackage[tight]{subfigure}
\usepackage{tikz}
\usepackage{pgfplots}
\usepackage{polski}
\usepackage[cp1250]{inputenc}
\usepackage{indentfirst}
\usepackage{color}
\usepackage{graphics}
\usepackage{geometry}
\usepackage{amsmath,amsfonts,amssymb}
% ---- PAGE LAYOUT ----
\geometry{a4paper, top=1.5cm, bottom=1.5cm, left=1.5cm,
right=1.5cm, nohead}
\begin{document}
\begin{center}
\large{\textbf{Egzamin z Analizy Matematycznej w przed�u�onej sesji poprawkowej (20/02/2023)}}
\end{center}

\noindent
\textbf{Zadanie 1} Oblicz ca�ki (a) $\int \frac{x^3 dx}{\sqrt{x^4+1}}$, (b) $\int x^4 \ln x dx$.\\


\noindent
\textbf{Zadanie 2} Zapisz wielomian Taylora stopnia II--go dla funkcji $y = \sqrt[3]{x}$ w okolicy $x_0 = 1$. Oblicz za jego pomoc� przybli�on� warto�� $\sqrt[3]{1{,}1}$.\\


\noindent
\textbf{Zadanie 3}  Wyznacz przedzia�y monotoniczno�ci i ekstrema lokalne funkcji: 
$y = 3 x^4+8 x^3-12 x^2-48 x$.\\


\noindent
\textbf{Zadanie 4} Wyznacz przybli�on� warto�� ca�ki $\int\limits_{0}^{3} x^3 dx$ dziel�c przedzia� ca�kowania na trzy r�wne cz�ci, za punkty po�rednie przyj�� �rodki kolejnych przedzia��w. Obliczenia prowadzi� na u�amkach zwyk�ych, wynik poda� w takiej samej postaci.\\

\noindent
\textbf{Zadanie 5}  Wyznacz pole obszaru ograniczonego liniami  $y = 4 - x^2$, $y = 3$. Wykonaj rysunek!\\

\noindent
\textbf{Zadanie 6}  Oblicz $z''_{xy}$, je�li $z = \ln (2x + 3xy - y^2)$.\\




\end{document}