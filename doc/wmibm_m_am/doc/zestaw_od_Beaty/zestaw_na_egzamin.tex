\documentclass[11pt,a4paper]{report}

\usepackage{float} %dzieki temu pakietowi rysunki pojawiaja sie tam gdzie chce po uzyciu opcji H w srodowisku figure
\usepackage[font=small,labelfont=bf]{caption}
\usepackage{natbib}
\usepackage{import}
\usepackage[tight]{subfigure}
\usepackage{tikz}
\usepackage{pgfplots}
\usepackage{polski}
\usepackage[cp1250]{inputenc}
\usepackage{indentfirst}
\usepackage{color}
\usepackage{graphics}
\usepackage{geometry}
\usepackage{amsmath,amsfonts,amssymb}
% ---- PAGE LAYOUT ----
\geometry{a4paper, top=1.5cm, bottom=1.5cm, left=1.5cm,
right=1.5cm, nohead}
\begin{document}
\begin{center}
\large{\textbf{Zestaw na egzamin}}
\end{center}

\noindent
\textbf{Zadanie 1} Oblicz pochodne: 
(a) $\left(\frac{1}{x^2} - \frac{1}{\sqrt[3]{x^2}}\right)'$,  
(b) $\left( \ln\frac{x^3 + 3x}{x^2 + 1} \right)'$.\\

\noindent
\textbf{Zadanie 2} Oblicz ca�ki (a) $\int \frac{x dx}{\sqrt{x^2+4}}$, (b) $\int x \cos x dx$.\\



\noindent
\textbf{Zadanie 3}  Zapisz wz�r Taylora dla funkcji $y = \ln x$ w okolicy $x_0 = 1$ z dok�adno�ci� do wyraz�w drugiego rz�du. Oblicz za pomoc� tego wzoru przybli�on� warto�� $\ln 0{,}9$.\\


\noindent
\textbf{Zadanie 4}  Wyznacz przedzia�y monotoniczno�ci i ekstrema lokalne funkcji: 
$y = x^4+2 x^3-4 x^2-12 x$.\\


\noindent
\textbf{Zadanie 5} Wyznacz przybli�on� warto�� ca�ki $\int\limits_{0}^{4} (4x-x^2) dx$ dziel�c przedzia� ca�kowania na cztery r�wne cz�ci, za punkty po�rednie przyj�� �rodki kolejnych przedzia��w. Obliczenia prowadzi� na u�amkach zwyk�ych, wynik poda� w takiej samej postaci.\\

\noindent
\textbf{Zadanie 6}  Wyznacz pole obszaru ograniczonego liniami  $y = x^2 + 2x$, $y = 4x + 3$. Wykonaj rysunek!\\

\noindent
\textbf{Zadanie 7}  Oblicz $z''_{xy}$, je�li $z = (x^2 + y^2 + 1)^2$.\\




\end{document}