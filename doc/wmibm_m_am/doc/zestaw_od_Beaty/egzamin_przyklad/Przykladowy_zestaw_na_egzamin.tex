\documentclass[11pt,a4paper]{report}

\usepackage{float} %dzieki temu pakietowi rysunki pojawiaja sie tam gdzie chce po uzyciu opcji H w srodowisku figure
\usepackage[font=small,labelfont=bf]{caption}
\usepackage{natbib}
\usepackage{import}
\usepackage[tight]{subfigure}
\usepackage{tikz}
\usepackage{pgfplots}
\usepackage{polski}
\usepackage[cp1250]{inputenc}
\usepackage{indentfirst}
\usepackage{color}
\usepackage{graphics}
\usepackage{geometry}
\usepackage{amsmath,amsfonts,amssymb}
% ---- PAGE LAYOUT ----
\geometry{a4paper, top=1.5cm, bottom=1.5cm, left=1.5cm,
right=1.5cm, nohead}
\begin{document}
\begin{center}
\large{\textbf{Przyk�adowy zestaw na egzamin z Analizy Matematycznej w przed�u�onej sesji poprawkowej}}
\end{center}

\noindent
\textbf{Zadanie 1} Oblicz pochodne: 
(a) $\left(\sqrt[3]{x^4} - \frac{1}{x}\right)'$,  
(b) $\left( x^3 e^{\cos x} \right)'$.\\

\noindent
\textbf{Zadanie 2} Oblicz ca�ki (a) $\int x^2 (1 + x^3)^4 dx$, (b) $\int \ln x dx$.\\



\noindent
\textbf{Zadanie 3} Zapisz wielomian Taylora stopnia II--go dla funkcji $y = \sqrt{x}$ w okolicy $x_0 = 4$. Oblicz za jego pomoc� przybli�on� warto�� $\sqrt{4{,}1}$.\\


\noindent
\textbf{Zadanie 4}  Wyznacz przedzia�y monotoniczno�ci i ekstrema lokalne funkcji: 
$y = 6 x^4-8 x^3-3 x^2+6 x$.\\


\noindent
\textbf{Zadanie 5} Wyznacz przybli�on� warto�� ca�ki $\int\limits_{-1}^{2} x^2 dx$ dziel�c przedzia� ca�kowania na \textbf{trzy} r�wne cz�ci, za punkty po�rednie przyj�� �rodki kolejnych przedzia��w. Obliczenia prowadzi� na u�amkach zwyk�ych, wynik poda� w takiej samej postaci.\\

\noindent
\textbf{Zadanie 6}  Wyznacz pole obszaru ograniczonego liniami  $y = 2x - x^2$, $y = 2 - x$. Wykonaj rysunek!\\

\noindent
\textbf{Zadanie 7}  Oblicz $z''_{xy}$, je�li $z = \ln (xy + 1)$.\\


\newpage

\begin{center}
	\large{\textbf{Odpowiedzi}}
\end{center}

\noindent
\textbf{Zadanie 1}\\
(a) $\left(\sqrt[3]{x^4} - \frac{1}{x}\right)' = \frac{1}{x^2} + \frac{4}{3}\sqrt[3]{x}$,\\  
(b) $\left( x^3 e^{\cos x} \right)' = 3x^2 e^{\cos x} - x^3 e^{\cos x}\sin x$.\\

\noindent
\textbf{Zadanie 2}\\
(a) $\int x^2 (1 + x^3)^4 dx = \frac{1}{15}(x^3 + 1)^{5} + C$, wskaz�wka: podstawi� $u = 1 + x^3$,\\ 
(b) $\int \ln x dx = x \ln x - x + C$, wskaz�wka: we wzorze na ca�kowanie przez cz�ci przyj��: $f(x) = \ln x$, $g'(x) = 1$.\\

\noindent
\textbf{Zadanie 3}\\
$W_2(x) = -\frac{1}{64} (x-4)^2+\frac{x-4}{4}+2$,\\
$\sqrt{4{,}1} \approx W_2(4{,}1) = -\frac{1}{64} (4{,}1-4)^2+\frac{4{,}1-4}{4}+2 = \ldots= \frac{12959}{6400}\approx 2{,}02485$.\\

\noindent
\textbf{Zadanie 4}\\
$y' = 24 x^3-24 x^2-6 x+6 = 6 (x-1) (2 x-1) (2 x+1)$,\\
funkcja ro�nie dla $x \in \left(-\frac 12, \frac 12\right),\ x\in \left( 1, +\infty\right)$,\\
funkcja maleje dla $x \in \left(-\infty, -\frac 12\right),\ x\in \left( \frac 12, 1\right)$,\\
minima lokalne dla $x = -\frac 12$ oraz dla $x = 1$,\\
maksimum lokalne dla $x = \frac 12$.\\

\noindent
\textbf{Zadanie 5}\\
przedzia�y $I_1 = \langle -1, 0 \rangle,\ I_2 = \langle 0, 1 \rangle,\ I_1 = \langle 1, 2 \rangle$, d�ugo�ci przedzia��w: $\Delta x_1 = \Delta x_2 = \Delta x_3 = 1$, punkty po�rednie: 
$c_1 = -\frac 12,\ c_2 = \frac 12,\ c_3 = \frac 32$,\\
przybli�enie ca�ki: $\int\limits_{-1}^{2} x^2 dx \approx f(c_1) \Delta x_1 + f(c_2) \Delta x_2 + f(c_3) \Delta x_3 = \ldots = \frac{11}{4}$.\\

\noindent
\textbf{Zadanie 6}\\
$S = \int\limits_{1}^{2} (2x-x^2 - (2-x)) dx = \ldots = \frac 16$.\\

\noindent
\textbf{Zadanie 7}\\
$z''_{xy} = \left(z'_y\right)'_x$, $z'_y = \frac{x}{x y+1}$, $z''_{xy} = \left(\frac{x}{x y+1}\right)'_{x} = \frac{1}{(x y+1)^2}$.

\end{document}