\documentclass{beamer}

\usetheme{Berlin}

\usepackage{polski}
\usepackage[cp1250]{inputenc}


\title{Egzamin z Matematyki 3 (WISGiE/I�/N, sesja poprawkowa)}
\date{12/02/2023}

\begin{document}
	
\frame{\titlepage}

\begin{frame}[t]
\frametitle{Zadanie 1 (0 - 10 pkt.)}

\noindent
\Large{
Oblicz pochodn� $z''_{xy}$ je�li 
$$z = \ln(xy - x^3)$$
}

\end{frame}

\begin{frame}[t]
\frametitle{Zadanie 2 (0 - 10 pkt.)}

\noindent
\Large{
Wyznacz ekstrema lokalne funkcji 
$$z = x^3-4 x y-3 x+y^2$$
}

\end{frame}

\begin{frame}[t]
\frametitle{Zadanie 3 (0 - 10 pkt.)}

\noindent
\Large{
Oblicz 
$$\iint\limits_{D} (x + 2y) dx dy,$$ 
gdzie $D$ -- tr�j�t $ABC$, przy czym $A(0,0)$, $(1,2)$, $C(1,4)$.
}
\end{frame}

\begin{frame}[t]
\frametitle{Zadanie 4 (0 - 10 pkt.)}

\noindent
\Large{
Oblicz
$$\iint\limits_{D}y dx dy$$ 
przechodz�c do wsp�rz�dnych biegunowych, gdzie 
$$D:\ x^2 + y^2 \geq 4,\ y\geq 0,\ x \geq 0$$
}

\end{frame}

\begin{frame}[t]
\frametitle{Zadanie 5 (0 - 10 pkt.)}

\noindent
\Large{
Rozwi�� r�wnanie r�niczkowe 
$$4\sqrt[3]{x^2} y' - \frac{1}{y^3} = 0,$$ 
uwzgl�dniaj�c warunek pocz�tkowy $y(1) = 1$.
}


\end{frame}

\begin{frame}[t]
\frametitle{Zadanie 6 (0 - 10 pkt.)}

\noindent
\Large{
Rozwi�� r�wnanie r�niczkowe 
$$
y'' - y' - 2y = 9e^{-x}
$$
}

\end{frame}

\begin{frame}[t]
\frametitle{Zadanie 7 (0 - 10 pkt.)}\


\noindent
\Large{
Na p�aszczy�nie $z = 2x - y + 1$ znale�� taki punkt $P$, dla kt�rego suma kwadrat�w wsp�rz�dnych jest minimalna.
}

\end{frame}

\begin{frame}[t]
\frametitle{Zadanie 8 (0 - 10 pkt.)}

\noindent
\Large{
W oparciu o ca�ki podw�jne wyznacz po�o�enie �rodka ci�ko�ci obszaru ograniczonegp liniami $y = -x$, $y = 3x$, $x = 1$.
}
\end{frame}


\end{document}