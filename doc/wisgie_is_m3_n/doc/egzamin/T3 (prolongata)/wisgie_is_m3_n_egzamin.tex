\documentclass[11pt,a4paper]{report}

\usepackage{float} %dzieki temu pakietowi rysunki pojawiaja sie tam gdzie chce po uzyciu opcji H w srodowisku figure
\usepackage[font=small,labelfont=bf]{caption}
\usepackage{natbib}
\usepackage{import}
\usepackage[tight]{subfigure}
\usepackage{tikz}
\usepackage{pgfplots}
\usepackage{polski}
\usepackage[cp1250]{inputenc}
\usepackage{indentfirst}
\usepackage{color}
\usepackage{graphics}
\usepackage{geometry}
\usepackage{amsmath,amsfonts,amssymb}
% ---- PAGE LAYOUT ----
\geometry{a4paper, top=1.5cm, bottom=1.5cm, left=1.5cm,
right=1.5cm, nohead}
\begin{document}
\begin{center}
\large{\textbf{Egzamin z matematyki 3 (WI�GiE/I�/N, przed�u�ona sesja poprawkowa), 19/02/2023}}
\end{center}

\noindent
\textbf{Zadanie 1 (0-10 pkt.)} Oblicz pochodn� $z''_{xy}$ je�li $z = \ln(x^2 - y^3)$.\\


\noindent
\textbf{Zadanie 2 (0-10 pkt.)} Wyznacz ekstrema lokalne funkcji $z = x^2 - 3xy + 4y^2 - 2x - 4y$.\\


\noindent
\textbf{Zadanie 3 (0-10 pkt.)}  Oblicz $\iint\limits_{D} (x + 2y) dx dy$, gdzie $D$ -- tr�j�t $ABC$, gdzie $A(0,0)$, $(1,2)$, $C(1,4)$.\\

\noindent
\textbf{Zadanie 4 (0-10 pkt.)}  Oblicz $\iint\limits_{D}x dx dy$ przechodz�c do wsp�rz�dnych biegunowych, gdzie $D$: $x^2 + y^2 \leq 1$,$x\geq 0$, $y \geq 0$.\\

\noindent
\textbf{Zadanie 5 (0-10 pkt.)}  Rozwi�� r�wnanie r�niczkowe $x^3 y' + 2 y^2 = 0$, uwzgl�dniaj�c warunek pocz�tkowy $y(1) = 1$.\\

\noindent
\textbf{Zadanie 6 (0-10 pkt.)}  Rozwi�� r�wnanie r�niczkowe $y'' - y' - 2y = 9e^{-x}$.\\ 



\end{document}