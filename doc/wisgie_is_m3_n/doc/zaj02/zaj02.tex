\documentclass[11pt,a4paper]{report}

\usepackage{float} %dzieki temu pakietowi rysunki pojawiaja sie tam gdzie chce po uzyciu opcji H w srodowisku figure
\usepackage[font=small,labelfont=bf]{caption}
\usepackage{natbib}
\usepackage{import}
\usepackage[tight]{subfigure}
\usepackage{tikz}
\usepackage{pgfplots}
\usepackage{polski}
\usepackage[cp1250]{inputenc}
\usepackage{indentfirst}
\usepackage{color}
\usepackage{graphics}
\usepackage{geometry}
\usepackage{amsmath,amssymb,amsfonts}
% ---- PAGE LAYOUT ----
\geometry{a4paper, top=1.5cm, bottom=1.5cm, left=1.5cm,
right=1.5cm, nohead}

\begin{document}

\begin{center}
	\large{\textbf{ZESTAW ZADA� II}}
\end{center}

\noindent
\textbf{Zadanie 1} W oparciu o r�niczk� zupe�n� oszacuj b��d z jakim wyznaczono pole prostok�ta $a\times b$, je�li dokonano nast�puj�cych pomiar�w bok�w: $a = 10\pm 0{,}01\text{ cm}$, 
$b = 20\pm 0{,}01\text{ cm}$.\\

\noindent
\textbf{Zadanie 2} Wyznaczy� ekstrema lokalne nast�puj�cych funkcji:\\
\textbf{(a)} $z = 10 x^2-14 x y+6 x+5 y^2-6 y-4$,
\textbf{(b)} $z = (x-3y+2)^2 +(y-x-2)^2$,\\
\textbf{(c)} $z = x^3-2 x y-x+y^2$, \textbf{(d)} $z = \left(x^2+y^2\right) e^{2 x}$.\\

\noindent
\textbf{Zadanie 3} Wyznacz wymiary otwartego pude�ka prostopad�o�ciennego o obj�to�ci $4\;\text{dm}^3$ aby jego pole powierzchni ca�kowitej by�o minimalne.\\

\noindent
\textbf{Zadanie 4} W oparciu o wyznaczenie ekstremum pewnej funkcji wyznacz odleg�o�� punktu $A(1,1,-4)$ od p�aszczyzny $x+y+z=1$.\\



\end{document}