\documentclass[11pt,a4paper]{report}

\usepackage{float} %dzieki temu pakietowi rysunki pojawiaja sie tam gdzie chce po uzyciu opcji H w srodowisku figure
\usepackage[font=small,labelfont=bf]{caption}
\usepackage{natbib}
\usepackage{import}
\usepackage[tight]{subfigure}
\usepackage{tikz}
\usepackage{pgfplots}
\usepackage{polski}
\usepackage[cp1250]{inputenc}
\usepackage{indentfirst}
\usepackage{color}
\usepackage{graphics}
\usepackage{geometry}
\usepackage{amsmath,amsfonts,amssymb}
% ---- PAGE LAYOUT ----
\geometry{a4paper, top=1.5cm, bottom=1.5cm, left=1.5cm,
right=1.5cm, nohead}
\begin{document}
	
\begin{center}
	\large{\textbf{ZESTAW ZADA� III}}
\end{center}

\noindent
\textbf{Zadanie 1} Naszkicuj obszary oraz zapisz je jako obszary normalne wzgl�dem osi $Ox$ lub $Oy$:\\
\textbf{(a)} tr�jk�t $ABC$, przy czym $A(0,0)$, $B(2,2)$ i $C(2,6)$, 
\textbf{(b)} obszar ograniczony liniami $y=x$, $x+y = 2$, $y = 2$.\\

\noindent
\textbf{Zadanie 2} Zamie� kolejno�� ca�kowania:\\
\textbf{(a)} $\int\limits_{-1}^{0}\left(\int\limits_{x+1}^{\sqrt{1-x^2}}f(x,y) dy\right)dx$,  
\textbf{(b)} $\int\limits_{0}^{1}\left(\int\limits_{-\sqrt{1-y^2}}^{1-y} f(x,y) dx \right)dy$.\\ 

\noindent
\textbf{Zadanie 3} Oblicz ca�ki podw�jne po podanych obszarach:\\
\textbf{(a)} $\iint\limits_D (x+y)dxdy$, $D$ -- tr�jk�t $ABC$, $A(0,0)$, $B(1,-1)$, $C(1,2)$,\\
\textbf{(b)} $\iint\limits_D (2x+3y) dxdy$, $D$ -- obszar ograniczony liniami $x+y=2$, $y=x$, $y=2$,\\
\textbf{(c)} $\iint\limits_D\frac{dxdy}{x^2+1}$, $D$ -- obszar ograniczony liniami $y = x$, $y = x^2$,\\
\textbf{(d)} $\iint\limits_D x dxdy$, $D$ -- obszar ograniczony okr�giem $x^2+y^2=9$, przy czym $y\geq 0$.\\



\end{document}