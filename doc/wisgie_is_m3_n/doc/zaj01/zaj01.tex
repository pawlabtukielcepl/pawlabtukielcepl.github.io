\documentclass[11pt,a4paper]{report}

\usepackage{float} %dzieki temu pakietowi rysunki pojawiaja sie tam gdzie chce po uzyciu opcji H w srodowisku figure
\usepackage[font=small,labelfont=bf]{caption}
\usepackage{natbib}
\usepackage{import}
\usepackage[tight]{subfigure}
\usepackage{tikz}
\usepackage{pgfplots}
\usepackage{polski}
\usepackage[cp1250]{inputenc}
\usepackage{indentfirst}
\usepackage{color}
\usepackage{graphics}
\usepackage{geometry}
\usepackage{amsmath,amssymb,amsfonts}
% ---- PAGE LAYOUT ----
\geometry{a4paper, top=1.5cm, bottom=1.5cm, left=1.5cm,
right=1.5cm, nohead}

\begin{document}

\begin{center}
\large{\textbf{ZESTAW ZADA� I}}
\end{center}

\noindent
\textbf{Zadanie 1} Oblicz pochodne cz�stkowe $\frac{\partial z}{\partial x}$ i $\frac{\partial z}{\partial y}$ nast�puj�cych funkcji:\\
\textbf{(a)} $z = x^2-3xy-y^2-4x+3y+5$, \textbf{(b)} $z = (2x-3y - 1)^2 + (3x+2y + 3)^2$, 
\textbf{(c)} $z = \ln(x^2 - 3xy)$,\\ 
\textbf{(d)} $z = x^2 e^{x^2 - 3y}$, 
\textbf{(e)} $z = \frac{2x+y}{x+2y}$, \textbf{(f)} $z = \arcsin\left(\frac{x}{xy+1}\right)$\\

\noindent
\textbf{Zadanie 2} Oblicz pochodne cz�stkowe drugiego rz�du $\frac{\partial^2 z}{\partial x^2}$, $\frac{\partial^2 z}{\partial y \partial x}$, 
$\frac{\partial^2 z}{\partial x \partial y}$ i $\frac{\partial^2 z}{\partial y^2}$ podanych funkcji:\\
\textbf{(a)} $z = x^2-3xy-4y^2+5y-3x+1$, \textbf{(b)} $z = \ln(xy-y^2)$, \textbf{(c)} $z = \frac{2x+3y}{3x+2y}$,\\ 
\textbf{(d)} $z = x \sin(x y)$, 
\textbf{(e)} $z = y^x$,  \textbf{(f)} $z = y\text{arctg}\,(x y)$\\



\end{document}