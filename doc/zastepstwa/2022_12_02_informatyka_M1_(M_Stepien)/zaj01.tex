\documentclass[11pt,a4paper]{report}

\usepackage{float} %dzieki temu pakietowi rysunki pojawiaja sie tam gdzie chce po uzyciu opcji H w srodowisku figure
\usepackage[font=small,labelfont=bf]{caption}
\usepackage{natbib}
\usepackage{import}
\usepackage[tight]{subfigure}
\usepackage{tikz}
\usepackage{pgfplots}
\usepackage{polski}
\usepackage[cp1250]{inputenc}
\usepackage{indentfirst}
\usepackage{color}
\usepackage{graphics}
\usepackage{geometry}
\usepackage{amsmath,amssymb,amsfonts}
% ---- PAGE LAYOUT ----
\geometry{a4paper, top=1.5cm, bottom=1.5cm, left=1.5cm,
right=1.5cm, nohead}

\begin{document}

\noindent
\textbf{Zadanie 1} \\
\textbf{(a)} zapisz r�wnanie prostej przechodz�cej przez punkty $A(3,-2,1)$ i $B(4,2,-1)$,\\
\textbf{(b)} zapisz r�wnanie og�lne p�aszczyzny przechodz�cej przez punkt $P(2,3,-1)$ i prostopad�ej do wektora $\vec{N} = [2,1,-3]$,\\
\textbf{(c)} zapisz r�wnanie og�lne p�aszczyzny przechodz�cej przez punkty $A(2,1,1)$, $B(-1,0,2)$ i $C(4,3,-2)$,\\
\textbf{(d)} zapisz r�wnanie prostej prostopad�ej do p�azzczyzny $3x+5y- 5z + 10 = 0$ i przechodz�cej przez punkt $P(2,2,-1)$,\\
\textbf{(e)} zapisz r�wnanie og�lne p�aszczyzny przechodz�cej przez punkt $P(2,2,0)$ i przez prost� $2x+4 = -3x+6 = -x + 1$.\\


\noindent
\textbf{Zadanie 2}\\ 
\textbf{(a)} wyznacz punkt przeci�cia prostych (o ile istnieje): $x=3t-5,\;y=-t+6,\;z=2t-3$ i $\frac{x-2}{-1}=\frac{y-2}{2}=\frac{z+1}{2}$\\
\textbf{(b)} znajd� punkt symetryczny do punktu $M=(8,2,-1)$ wzgl�dem p�aszczyzny $6x-2y+z-2=0$,\\
\textbf{(c)} znajd� punkt symetryczny do punktu $M=(5,1,-2)$ wzgl�dem prostej $\frac{x-4}{3}=\frac{y}{2}=\frac{z+7}{-1}$.\\




\end{document}