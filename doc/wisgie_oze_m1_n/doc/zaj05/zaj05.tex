\documentclass[11pt,a4paper]{report}

\usepackage{float} %dzieki temu pakietowi rysunki pojawiaja sie tam gdzie chce po uzyciu opcji H w srodowisku figure
\usepackage[font=small,labelfont=bf]{caption}
\usepackage{natbib}
\usepackage{import}
\usepackage[tight]{subfigure}
\usepackage{tikz}
\usepackage{pgfplots}
\usepackage{polski}
\usepackage[cp1250]{inputenc}
\usepackage{indentfirst}
\usepackage{color}
\usepackage{graphics}
\usepackage{geometry}
\usepackage{amsmath,amssymb,amsfonts}
% ---- PAGE LAYOUT ----
\geometry{a4paper, top=1.5cm, bottom=1.5cm, left=1.5cm,
right=1.5cm, nohead}

\begin{document}

\begin{center}
\large{\textbf{ZESTAW ZADA� V}}
\end{center}

\noindent
\textbf{Zadanie 1} Oblicz ca�ki nieoznaczone przez podstawienie:\\
\textbf{(a)} $\int e^{-3x}dx$, \textbf{(b)} $\int\sqrt{3x+4}dx$, \textbf{(c)} $\int x(3x^2+1)^5dx$, \textbf{(d)} $\int x^2\sin(1-x^3)dx$, \textbf{(e)} $\int \frac{dx}{x\ln^2 x}$,\\
\textbf{(f)} $\int \frac{\cos x dx}{1 + \sin^2 x}$, \textbf{(g)} $\int \sin^3 x dx$ , \textbf{(h)} $\int \frac{e^{\frac 1 x}dx}{x^2}$, \textbf{(i)} $\int\frac{e^x dx}{\sqrt{1-e^{2x}}}$,\\
\textbf{(j)} $\int \cos^2 x dx$, \textbf{(k)} $\int \frac{\arcsin^3 x dx}{\sqrt{1-x^2}}$, \textbf{(l)} $\int(x+2)\sqrt{x-1}dx$, \textbf{(m)} $\int \frac{x^2 dx}{x^6 + 1}$,\\ 
\textbf{(n)} $\int\frac{dx}{\sqrt{x}\sqrt{1-x}}$, \textbf{(o)} $\int\sqrt{4-x^2}dx$.\\ 

\end{document}