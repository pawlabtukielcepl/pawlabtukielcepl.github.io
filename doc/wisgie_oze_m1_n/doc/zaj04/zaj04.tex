\documentclass[11pt,a4paper]{report}

\usepackage{float} %dzieki temu pakietowi rysunki pojawiaja sie tam gdzie chce po uzyciu opcji H w srodowisku figure
\usepackage[font=small,labelfont=bf]{caption}
\usepackage{natbib}
\usepackage{import}
\usepackage[tight]{subfigure}
\usepackage{tikz}
\usepackage{pgfplots}
\usepackage{polski}
\usepackage[cp1250]{inputenc}
\usepackage{indentfirst}
\usepackage{color}
\usepackage{graphics}
\usepackage{geometry}
\usepackage{amsmath,amssymb,amsfonts}
% ---- PAGE LAYOUT ----
\geometry{a4paper, top=1.5cm, bottom=1.5cm, left=1.5cm,
right=1.5cm, nohead}

\begin{document}

\begin{center}
\large{\textbf{ZESTAW ZADA� IV}}
\end{center}

\noindent
\textbf{Zadanie 1} Zapisz liczby w postaci $a+bi$, gdzie $a,b$ -- liczby rzeczywiste, $i$ -- jednostka urojona zdefiniowana r�wnaniem $i^2 = -1$:\\ 
\textbf{(a)} $i + 2i^3 + 3i^6 + 4i^9 + 5i^{12}$, \textbf{(b)} $(2+3i)^2$, \textbf{(c)} $(1-i)(3+i)$, \textbf{(d)} $(1+i)^2-(3+i)^3$,\\ 
\textbf{(e)} $\frac{1}{i} + \frac{2}{i^3} + \frac{3}{i^5} + \frac{4}{i^7} + \frac{5}{i^9}$,  \textbf{(f)} $\frac{1-i}{1+i}$, 
\textbf{(g)} $\frac{4+3 i}{2-i}+(1-3 i)\cdot(-2+2 i)$, \textbf{(h)} $\frac{1}{1+2i} - \frac{2+i}{-1+3i}$.\\

\noindent
\textbf{Zadanie 2} Rozwi�� r�wnania w dziedzinie zespolonej:\\
\textbf{(a)} $z^2-4 z+13 = 0$, \textbf{(b)} $z^4+7 z^2+12 = 0$,  
\textbf{(c)} $2 z^3-3 z^2+8 z-12 = 0$, \textbf{(d)} $z^2 + 8 - 6i = 0$, 
\textbf{(e)} $z^3 - 27 = 0$.\\

\noindent
\textbf{Zadanie 3} Oblicz ca�ki nieoznaczone:\\
\textbf{(a)} $\int (4x^2 - 3x + 5)dx$,  
\textbf{(b)} $\int\left(\frac{1}{x^2} - \frac{2}{x^3} + \frac{3}{x^4} - \frac{5}{x^5}\right)dx$,
\textbf{(c)} $\int\left(3\sqrt[3]{x} - \frac{2}{\sqrt[3]{x}}\right)dx$,\\
\textbf{(d)} $\int\left(2\sin x + 3\cos x \right)dx$, 
\textbf{(e)} $\int\left(\frac{2}{\cos^2 x} - \frac{5}{\sin^2 x}\right)dx$, 
\textbf{(f)} $\int\left(\frac{3}{\sqrt{1-x^2}} - \frac{4}{x^2 + 1}\right)dx$,\\
\textbf{(g)} $\int\left(3e^x - \frac{5}{x}\right)dx$, 
\textbf{(h)} $\int\cos(3x)dx$, 
\textbf{(i)} $\int e^{2x}\cos(3x)dx$.\\

\noindent
\textbf{Zadanie 4} Oblicz ca�ki nieoznaczone stosuj�c podane podstawienia:\\ 
\textbf{(a)} $\int e^{2x} dx$, $u = 2x$, \textbf{(b)} $\int \frac{e^{x} dx}{(3e^x - 2)^5}$, $u = 3e^x - 2$, \textbf{(c)} $\int x^3 (x^4 + 1)^{99} dx$, $u = x^4 + 1$,\\ 
\textbf{(d)} $\int \frac{\cos x dx}{\sin^3 x}$, $u = \sin x$, \textbf{(e)} $\int x^2 \sqrt{x + 1} dx$, $u = x + 1$.\\


\end{document}