\documentclass[11pt,a4paper]{report}

\usepackage{float} %dzieki temu pakietowi rysunki pojawiaja sie tam gdzie chce po uzyciu opcji H w srodowisku figure
\usepackage[font=small,labelfont=bf]{caption}
\usepackage{natbib}
\usepackage{import}
\usepackage[tight]{subfigure}
\usepackage{tikz}
\usepackage{pgfplots}
\usepackage{polski}
\usepackage[cp1250]{inputenc}
\usepackage{indentfirst}
\usepackage{color}
\usepackage{graphics}
\usepackage{geometry}
\usepackage{amsmath,amssymb,amsfonts}
% ---- PAGE LAYOUT ----
\geometry{a4paper, top=1.5cm, bottom=1.5cm, left=1.5cm,
right=1.5cm, nohead}

\begin{document}

\begin{center}
\large{\textbf{ZESTAW ZADA� VIII}}
\end{center}

\noindent
\textbf{Zadanie 1} Dane s� macierze:
$$
A = \left [
\begin{array}{cc}
	2 & 3\\
	1 & 2\\
\end{array}
\right],\ \ \ 
B = \left [
\begin{array}{cc}
	1 & 2\\
	-1 & 1\\
\end{array}
\right],\ \ \ 
C = \left [
\begin{array}{cc}
	0 & 1\\
	3 & -2\\
	3 & 1\\
\end{array}
\right],\ \ \ 
D = \left [
\begin{array}{ccc}
	2 & 1 & 1\\
	1 & 2 & 2\\
\end{array}
\right] 
$$
Oblicz:
\textbf{(a)} $(3A-2B^T)^T$, \textbf{(b)} $(A^2 - D\cdot C)^T$.\\


\noindent
\textbf{Zadanie 2} Rozwi�� uk�ady r�wna� metod� Gaussa eliminacji:
$$
\textbf{(a)}\ 
\left \{
\begin{array}{l}
	x - 3y + 2z = 3\\
	2x + y - 3z = -1\\
	3x + 2y + z = 4
\end{array}
\right .,\ \ \ 
\textbf{(b)}\ 
\left \{
\begin{array}{l}
	2x - y + 3z = 0\\
	x  - 2z = 1\\
	x - y + 5z = -1
\end{array}
\right .,\ \ \ 
\textbf{(c)}\ 
\left \{
\begin{array}{l}
	x - y + z = 1\\
	2x + y - t = 2\\
	-2x + 3z + t = 2\\
	3y - z + 2t = 4\\
\end{array}
\right .,\ \ \ 
$$





\end{document}