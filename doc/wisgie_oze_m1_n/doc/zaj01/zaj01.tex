\documentclass[11pt,a4paper]{report}

\usepackage{float} %dzieki temu pakietowi rysunki pojawiaja sie tam gdzie chce po uzyciu opcji H w srodowisku figure
\usepackage[font=small,labelfont=bf]{caption}
\usepackage{natbib}
\usepackage{import}
\usepackage[tight]{subfigure}
\usepackage{tikz}
\usepackage{pgfplots}
\usepackage{polski}
\usepackage[cp1250]{inputenc}
\usepackage{indentfirst}
\usepackage{color}
\usepackage{graphics}
\usepackage{geometry}
\usepackage{amsmath,amssymb,amsfonts}
% ---- PAGE LAYOUT ----
\geometry{a4paper, top=1.5cm, bottom=1.5cm, left=1.5cm,
right=1.5cm, nohead}

\begin{document}

\begin{center}
\large{\textbf{ZESTAW ZADA� I}}
\end{center}

\noindent
\textbf{Zadanie 1} Oblicz granice:\\ 
\textbf{(a)} $\lim\limits_{x\rightarrow \frac 23} (x^2-5x+6)$, 
\textbf{(b)} $\lim\limits_{x\rightarrow 3}\frac{x^2-9}{x-3}$, 
\textbf{(c)} $\lim\limits_{x\rightarrow 1}\frac{x^2+3x-4}{x^2+2x-3}$, 
\textbf{(d)} $\lim\limits_{x\rightarrow 0}\left(\frac{1}{x}+\frac{2}{x^2-2x}\right)$, 
\textbf{(e)} $\lim\limits_{x\rightarrow 0}\frac{\sqrt{x+1}-1}{\sqrt{x+4}-2}$.\\

\noindent
\textbf{Zadanie 2}\\
\textbf{(a)} W oparciu o definicj� oblicz pochodn� funkcji $f(x) = 2x^2 + 3x + 4$ w punkcie $x_0 = -1$, zapisz r�wnanie stycznej do wykresu funkcji w punkcie $(-1, 3)$,\\
\textbf{(b)} W oparciu o definicj� wyprowad� wz�r na pochodn� funkcji $f(x) = \frac{1}{x}$,\\
\textbf{(c)} W oparciu o regu�� r�niczkowania iloczynu i wz�r uzyskany w (b) wyprowad� wz�r na pochodn� funkcji $f(x) = \frac{1}{\sqrt{x}}$.\\


\noindent
\textbf{Zadanie 3} Oblicz pochodne:\\
\textbf{(a)} $\left(x^6 - 3x^4 +5x^3 - 6x - 5\right)'$, 
\textbf{(b)} $\left(\frac{1}{x} - \sqrt{x} + \frac{10}{\sqrt{x}}\right)'$, 
\textbf{(c)} $\left((-5x^2 + 3x + 2)\cdot (x^2 - 3x + 1)\right)'$, 
\textbf{(d)} $\left(\frac{1}{x^3+1}\right)'$, 
\textbf{(e)} $\left(\frac{x^2 - 3x + 2}{x^2 - 1}\right)'$. 





\end{document}