\documentclass{beamer}

\usetheme{Berlin}

\usepackage{polski}
\usepackage[cp1250]{inputenc}


\title{Egzamin z Matematyki 1 (WISGiE/OZE/N, przed�u�ona sesja poprawkowa)}
\date{19/02/2023}

\begin{document}
	
\frame{\titlepage}

\begin{frame}[t]
\frametitle{Zadanie 1 (0 - 10 pkt.)}

\noindent
\Large{
Oblicz pochodne: 
$$\left(\frac{1}{x^2} + 2\sqrt{x}\right)'$$ 
$$\left( \frac{\text{arctg}\,x}{\ln x} \right)'$$
$$\left( x^3\,\text{tg}\,(x^2) \right)'$$
}

\end{frame}

\begin{frame}[t]
\frametitle{Zadanie 2 (0 - 10 pkt.)}

\noindent
\Large{
Wyznacz przedzia�y monotoniczno�ci i ekstrema lokalne funkcji: 
$$y = 3 x^4-4 x^3-24 x^2+48 x$$
}

\end{frame}

\begin{frame}[t]
\frametitle{Zadanie 3 (0 - 10 pkt.)}

\noindent
\Large{
(a) Zapisz liczb� $z = \frac{1}{2+3i} - \frac{i}{3 - 2i}$ w postaci $a + bi$, gdzie $a,b$ -- liczby rzeczywiste.\\ 
(b) Rozwi�� r�wnanie $z^2 + 4z + 29 = 0$ w dziedzinie zespolonej
}
\end{frame}

\begin{frame}[t]
\frametitle{Zadanie 4 (0 - 10 pkt.)}

\noindent
\Large{
Oblicz ca�k�: 
$$\int \frac{4x + 7}{x^2 + x - 6} dx$$
}

\end{frame}

\begin{frame}[t]
\frametitle{Zadanie 5 (0 - 10 pkt.)}

\noindent
\Large{
Wyznacz pole obszaru ograniczonego liniami  
$$y = x^2,\ \ y = x + 2$$ 
Wykonaj rysunek!
}


\end{frame}

\begin{frame}[t]
\frametitle{Zadanie 6 (0 - 10 pkt.)}
\noindent
\Large{
Rozwi�� uk�ad r�wna� wybran� metod� ( tzn. metod� Gaussa eliminacji lub w oparciu o wzory Cramera):
$$
\left\{
\begin{array}{l}
	x + 2y + 3z = 2\\
	2x + y - z = 5\\
	-3x + 2y + 3z = -2\\
\end{array}
\right.
$$
}
\end{frame}


\end{document}