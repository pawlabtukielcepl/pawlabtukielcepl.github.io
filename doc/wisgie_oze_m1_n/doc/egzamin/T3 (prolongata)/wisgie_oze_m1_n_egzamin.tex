\documentclass[11pt,a4paper]{report}

\usepackage{float} %dzieki temu pakietowi rysunki pojawiaja sie tam gdzie chce po uzyciu opcji H w srodowisku figure
\usepackage[font=small,labelfont=bf]{caption}
\usepackage{natbib}
\usepackage{import}
\usepackage[tight]{subfigure}
\usepackage{tikz}
\usepackage{pgfplots}
\usepackage{polski}
\usepackage[cp1250]{inputenc}
\usepackage{indentfirst}
\usepackage{color}
\usepackage{graphics}
\usepackage{geometry}
\usepackage{amsmath,amsfonts,amssymb}
% ---- PAGE LAYOUT ----
\geometry{a4paper, top=1.5cm, bottom=1.5cm, left=1.5cm,
right=1.5cm, nohead}
\begin{document}
\begin{center}
\large{\textbf{Egzamin z matematyki 1 (WI�GiE/OZE/N, przed�u�ona sesja poprawkowa), 19/02/2023}}
\end{center}

\noindent
\textbf{Zadanie 1 (0-10 pkt.)} Oblicz pochodne: 
$\left(\frac{1}{x^2} + 2\sqrt{x}\right)'$, 
$\left( \frac{\text{arctg}\,x}{\ln x} \right)'$, 
$\left( x^3\,\text{tg}\,(x^2) \right)'$.\\


\noindent
\textbf{Zadanie 2 (0-10 pkt.)}  Wyznacz przedzia�y monotoniczno�ci i ekstrema lokalne funkcji: 
$y = 3 x^4-4 x^3-24 x^2+48 x$.\\


\noindent
\textbf{Zadanie 3 (0-10 pkt.)}  (a) Zapisz liczb� $z = \frac{1}{2+3i} - \frac{i}{3 - 2i}$ w postaci $a + bi$, gdzie $a,b$ -- liczby rzeczywiste. (b) Rozwi�� r�wnanie $z^2 + 4z + 29 = 0$ w dziedzinie zespolonej.\\

\noindent
\textbf{Zadanie 4 (0-10 pkt.)}  Oblicz ca�k�: 
$\int \frac{4x + 7}{x^2 + x - 6} dx$.\\

\noindent
\textbf{Zadanie 5 (0-10 pkt.)}  Wyznacz pole obszaru ograniczonego liniami  $y = x^2$, $y = x + 2$. Wykonaj rysunek!\\

\noindent
\textbf{Zadanie 6 (0-10 pkt.)} Rozwi�� uk�ad r�wna� wybran� metod� (tzn. metod� Gaussa eliminacji lub w oparciu o wzory Cramera):
$$
\left\{
\begin{array}{l}
	x + 2y + 3z = 2\\
	2x + y - z = 5\\
	-3x + 2y + 3z = -2\\
\end{array}
\right.
$$ 

\end{document}