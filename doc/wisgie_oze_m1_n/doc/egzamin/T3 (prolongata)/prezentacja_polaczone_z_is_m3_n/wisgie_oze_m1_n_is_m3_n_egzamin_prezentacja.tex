\documentclass{beamer}

\usetheme{Berlin}

\usepackage{polski}
\usepackage[cp1250]{inputenc}


\title{Egzamin z Matematyki 1/Matematyki 3 (WISGiE/OZE/N/I�/N, przed�u�ona sesja poprawkowa)}
\date{19/02/2023}

\begin{document}
	
\frame{\titlepage}

\begin{frame}[t]
\frametitle{Zadanie 1 (0 - 10 pkt.)}


\Large{
\noindent
\textbf{OZE:} 
Oblicz pochodne: 
$$\left(\frac{1}{x^2} + 2\sqrt{x}\right)'$$ 
$$\left( \frac{\text{arctg}\,x}{\ln x} \right)'$$
$$\left( x^3\,\text{tg}\,(x^2) \right)'$$
}

\Large{
	\noindent
	\textbf{I�:} 
	Oblicz pochodn� $z''_{xy}$ je�li 
	$$z = \ln(x^2 - y^3)$$
}

\end{frame}

\begin{frame}[t]
\frametitle{Zadanie 2 (0 - 10 pkt.)}

\noindent
\Large{
\noindent
\textbf{OZE:} 
Wyznacz przedzia�y monotoniczno�ci i ekstrema lokalne funkcji: 
$$y = 3 x^4-4 x^3-24 x^2+48 x$$
}

\noindent
\Large{
	\noindent
	\textbf{I�:} 
	Wyznacz ekstrema lokalne funkcji 
	$$z = x^2 - 3xy + 4y^2 - 2x - 4y$$
}

\end{frame}

\begin{frame}[t]
\frametitle{Zadanie 3 (0 - 10 pkt.)}

\noindent
\Large{
\noindent
\textbf{OZE:} 
(a) Zapisz liczb� $z = \frac{1}{2+3i} - \frac{i}{3 - 2i}$ w postaci $a + bi$, gdzie $a,b$ -- liczby rzeczywiste.\\ 
(b) Rozwi�� r�wnanie $z^2 + 4z + 29 = 0$ w dziedzinie zespolonej.
}

\Large{
	\noindent
	\textbf{I�:} 
	Oblicz 
	$$\iint\limits_{D} (x + 2y) dx dy,$$ 
	gdzie $D$ -- tr�j�t $ABC$, przy czym $A(0,0)$, $(1,2)$, $C(1,4)$.
}
\end{frame}

\begin{frame}[t]
\frametitle{Zadanie 4 (0 - 10 pkt.)}

\noindent
\Large{
	\noindent
	\textbf{OZE:} 
Oblicz ca�k�: 
$$\int \frac{4x + 7}{x^2 + x - 6} dx$$
}

\Large{
	\noindent
	\textbf{I�:} 
Oblicz
$$\iint\limits_{D} x dx dy$$ 
przechodz�c do wsp�rz�dnych biegunowych, gdzie 
$$D:\ x^2 + y^2 \leq 1,\ x\geq 0,\ y \geq 0$$
}

\end{frame}

\begin{frame}[t]
\frametitle{Zadanie 5 (0 - 10 pkt.)}

\noindent
\Large{
	\noindent
	\textbf{OZE:} 
Wyznacz pole obszaru ograniczonego liniami  
$$y = x^2,\ \ y = x + 2$$ 
Wykonaj rysunek!
}

\noindent
\Large{
	\noindent
	\textbf{I�:} 
Rozwi�� r�wnanie r�niczkowe 
$$x^3 y' + 2 y^2 = 0,$$ 
uwzgl�dniaj�c warunek pocz�tkowy $y(1) = 1$.
}


\end{frame}

\begin{frame}[t]
\frametitle{Zadanie 6 (0 - 10 pkt.)}

\noindent
\Large{
	\noindent
	\textbf{OZE:} 
Rozwi�� uk�ad r�wna� wybran� metod� ( tzn. metod� Gaussa eliminacji lub w oparciu o wzory Cramera):
$$
\left\{
\begin{array}{l}
	x + 2y + 3z = 2\\
	2x + y - z = 5\\
	-3x + 2y + 3z = -2\\
\end{array}
\right.
$$
}

\Large{
	\noindent
	\textbf{I�:} 
Rozwi�� r�wnanie r�niczkowe 
$$
y'' - y' - 2y = 9e^{-x}
$$
}

\end{frame}

\end{document}