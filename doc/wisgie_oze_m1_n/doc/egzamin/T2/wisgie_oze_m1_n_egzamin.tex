\documentclass[11pt,a4paper]{report}

\usepackage{float} %dzieki temu pakietowi rysunki pojawiaja sie tam gdzie chce po uzyciu opcji H w srodowisku figure
\usepackage[font=small,labelfont=bf]{caption}
\usepackage{natbib}
\usepackage{import}
\usepackage[tight]{subfigure}
\usepackage{tikz}
\usepackage{pgfplots}
\usepackage{polski}
\usepackage[cp1250]{inputenc}
\usepackage{indentfirst}
\usepackage{color}
\usepackage{graphics}
\usepackage{geometry}
\usepackage{amsmath,amsfonts,amssymb}
% ---- PAGE LAYOUT ----
\geometry{a4paper, top=1.5cm, bottom=1.5cm, left=1.5cm,
right=1.5cm, nohead}
\begin{document}
\begin{center}
\large{\textbf{Egzamin z matematyki 1 (WI�GiE/OZE/N, sesja poprawkowa), 12/02/2023}}
\end{center}

\noindent
\textbf{Zadanie 1 (0-10 pkt.)} Oblicz pochodne: 
$\left(\frac{1}{x} + 3\sqrt[3]{x^2}\right)'$, 
$\left( \frac{\text{tg}\,x}{\sin x} \right)'$, 
$\left( x^3\sin(x^2) \right)'$.\\


\noindent
\textbf{Zadanie 2 (0-10 pkt.)}  Wyznacz przedzia�y monotoniczno�ci i ekstrema lokalne funkcji: 
$y = 3 x^4-4 x^3-24 x^2+48 x$.\\


\noindent
\textbf{Zadanie 3 (0-10 pkt.)}  (a) Zapisz liczb� $z = \frac{3-2i}{1+2i} + \frac{2}{i^3}$ w postaci $a + bi$, gdzie $a,b$ -- liczby rzeczywiste. (b) Rozwi�� r�wnanie $z^2-2 z+2 = 0$ w dziedzinie zespolonej.\\

\noindent
\textbf{Zadanie 4 (0-10 pkt.)}  Oblicz ca�k�: 
$\int \frac{4 x+1}{x^2-x-2} dx$.\\

\noindent
\textbf{Zadanie 5 (0-10 pkt.)}  Wyznacz pole obszaru ograniczonego liniami  $y = x^2$, $y = x + 2$. Wykonaj rysunek!\\

\noindent
\textbf{Zadanie 6 (0-10 pkt.)} Rozwi�� uk�ad r�wna� wybran� metod� ( tzn. metod� Gaussa eliminacji lub w oparciu o wzory Cramera):
$$
\left\{
\begin{array}{l}
	2x + x + z = 3\\
	-x +  y + 2z = -2\\
	3x + 2y - z = 1\\
\end{array}
\right.
$$ 

\noindent
\textbf{Zadanie 7 (0-10 pkt.)}  W oparciu o definicj� oblicz pochodn� podanej funkcji $f(x) = 5x^2-7x-6$ w punkcie $x_0 = 1$. Zapisz r�wnanie stycznej do wykresu funkcji w punkcie $(x_0,f(x_0))$, naszkicuj pogl�dowy wykres funkcji oraz stycznej.\\ 

\noindent
\textbf{Zadanie 8 (0-10 pkt.)}\\ 
W oparciu o rachunek ca�kowy wyznacz po�o�enie �rodka ci�ko�ci obszaru ograniczonego liniami $y = -x$, $y = 2x$, $x = 2$ (pole obszaru oblicz za pomoc� wzoru na pole tr�jk�ta).

\end{document}