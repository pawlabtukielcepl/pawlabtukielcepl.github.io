\documentclass{beamer}

\usetheme{Berlin}

\usepackage{polski}
\usepackage[cp1250]{inputenc}


\title{Egzamin z Matematyki 1 (WISGiE/OZE/N, termin pierwszy)}
\date{12/02/2023}

\begin{document}
	
\frame{\titlepage}

\begin{frame}[t]
\frametitle{Zadanie 1 (0 - 10 pkt.)}

\noindent
\Large{
Oblicz pochodne: 
$$\left(\frac{1}{x} + 3\sqrt[3]{x^2}\right)'$$ 
$$\left( \frac{\text{tg}\,x}{\sin x} \right)'$$ 
$$\left( x^3\sin(x^2) \right)'$$
}

\end{frame}

\begin{frame}[t]
\frametitle{Zadanie 2 (0 - 10 pkt.)}

\noindent
\Large{
Wyznacz przedzia�y monotoniczno�ci i ekstrema lokalne funkcji: 
$$y = 3 x^4-4 x^3-24 x^2+48 x$$
}

\end{frame}

\begin{frame}[t]
\frametitle{Zadanie 3 (0 - 10 pkt.)}

\noindent
\Large{
(a) Zapisz liczb� $z = \frac{3-2i}{1+2i} + \frac{2}{i^3}$ w postaci $a + bi$, gdzie $a,b$ -- liczby rzeczywiste.\\ 
(b) Rozwi�� r�wnanie $z^2-2 z+2 = 0$ w dziedzinie zespolonej.\\
}
\end{frame}

\begin{frame}[t]
\frametitle{Zadanie 4 (0 - 10 pkt.)}

\noindent
\Large{
Oblicz ca�k�: 
$$\int \frac{4 x+1}{x^2-x-2} dx$$
}

\end{frame}

\begin{frame}[t]
\frametitle{Zadanie 5 (0 - 10 pkt.)}

\noindent
\Large{
Wyznacz pole obszaru ograniczonego liniami  
$$y = x^2,\ \ y = x + 2$$ 
Wykonaj rysunek!
}


\end{frame}

\begin{frame}[t]
\frametitle{Zadanie 6 (0 - 10 pkt.)}

\noindent
\Large{
Rozwi�� uk�ad r�wna� wybran� metod� ( tzn. metod� Gaussa eliminacji lub w oparciu o wzory Cramera):
$$
\left\{
\begin{array}{l}
	2x + x + z = 3\\
	-x +  y + 2z = -2\\
	3x + 2y - z = 1\\
\end{array}
\right.
$$ 
}

\end{frame}

\begin{frame}[t]
\frametitle{Zadanie 7 (0 - 10 pkt.)}\


\noindent
\Large{
W oparciu o definicj� oblicz pochodn� podanej funkcji 
$$f(x) = 5x^2-7x-6$$ 
w punkcie $x_0 = 1$. Zapisz r�wnanie stycznej do wykresu funkcji w punkcie $(x_0,f(x_0))$, naszkicuj pogl�dowy wykres funkcji oraz stycznej.
}

\end{frame}

\begin{frame}[t]
\frametitle{Zadanie 8 (0 - 10 pkt.)}

\noindent
\Large{
W oparciu o rachunek ca�kowy wyznacz po�o�enie �rodka ci�ko�ci obszaru ograniczonego liniami $y = -x$, $y = 2x$, $x = 2$ (pole obszaru oblicz za pomoc� wzoru na pole tr�jk�ta).
}
\end{frame}


\end{document}