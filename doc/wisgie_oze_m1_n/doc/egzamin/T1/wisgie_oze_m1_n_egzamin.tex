\documentclass[11pt,a4paper]{report}

\usepackage{float} %dzieki temu pakietowi rysunki pojawiaja sie tam gdzie chce po uzyciu opcji H w srodowisku figure
\usepackage[font=small,labelfont=bf]{caption}
\usepackage{natbib}
\usepackage{import}
\usepackage[tight]{subfigure}
\usepackage{tikz}
\usepackage{pgfplots}
\usepackage{polski}
\usepackage[cp1250]{inputenc}
\usepackage{indentfirst}
\usepackage{color}
\usepackage{graphics}
\usepackage{geometry}
\usepackage{amsmath,amsfonts,amssymb}
% ---- PAGE LAYOUT ----
\geometry{a4paper, top=1.5cm, bottom=1.5cm, left=1.5cm,
right=1.5cm, nohead}
\begin{document}
\begin{center}
\large{\textbf{Egzamin z matematyki 1 (WI�GiE/OZE, termin pierwszy), 05/02/2023}}
\end{center}

\noindent
\textbf{Zadanie 1 (0-10 pkt.)} Oblicz pochodne: 
$\left(\frac{5}{x^3} - \frac{4}{\sqrt[4]{x^7}}\right)'$, 
$\left( \frac{\sin x}{\ln x} \right)'$, 
$\left( \text{tg}\,(3x) e^{x^3} \right)'$.\\


\noindent
\textbf{Zadanie 2 (0-10 pkt.)}  Wyznacz przedzia�y monotoniczno�ci i ekstrema lokalne funkcji: 
$y = 6 x^4+8 x^3-3 x^2-6 x$.\\


\noindent
\textbf{Zadanie 3 (0-10 pkt.)}  (a) Zapisz liczb� $z = \frac{2-5i}{5+2i} + i^{28}$ w postaci $a + bi$, gdzie $a,b$ -- liczby rzeczywiste. (b) Rozwi�� r�wnanie $z^2 - 6z + 13 = 0$ w dziedzinie zespolonej.\\

\noindent
\textbf{Zadanie 4 (0-10 pkt.)}  Oblicz ca�k�: 
$\int \frac{7 x-1}{x^2-x-6} dx$.\\

\noindent
\textbf{Zadanie 5 (0-10 pkt.)}  Wyznacz pole obszaru ograniczonego liniami  $y = x^2 - 2x$, $y = x - 2$. Wykonaj rysunek!\\

\noindent
\textbf{Zadanie 6 (0-10 pkt.)} Rozwi�� uk�ad r�wna� metod� Gaussa eliminacji:
$$
\left\{
\begin{array}{l}
x + 2x - z = -1\\
2x + 3y - z = 0\\
-3x + 2y + z = -1\\
\end{array}
\right.
$$ 

\noindent
\textbf{Zadanie 7 (0-20 pkt.)}  W oparciu o definicj� oblicz pochodn� podanej funkcji $f(x) = 2x^2-3x+1$ w punkcie $x_0 = 1$. Zapisz r�wnanie stycznej do wykresu funkcji w punkcie $(x_0,f(x_0))$, naszkicuj pogl�dowy wykres funkcji oraz stycznej.\\ 

\noindent
\textbf{Zadanie 8 (0-20 pkt.)}\\ 
W oparciu o rachunek ca�kowy wyznacz po�o�enie �rodka ci�ko�ci obszaru ograniczonego liniami $y = x^2$, $y = 2x$, je�li wiadomo, �e pole obszaru wynosi $S = \frac 43$.






\end{document}