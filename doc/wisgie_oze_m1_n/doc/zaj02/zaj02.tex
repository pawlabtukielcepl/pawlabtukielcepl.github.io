\documentclass[11pt,a4paper]{report}

\usepackage{float} %dzieki temu pakietowi rysunki pojawiaja sie tam gdzie chce po uzyciu opcji H w srodowisku figure
\usepackage[font=small,labelfont=bf]{caption}
\usepackage{natbib}
\usepackage{import}
\usepackage[tight]{subfigure}
\usepackage{tikz}
\usepackage{pgfplots}
\usepackage{polski}
\usepackage[cp1250]{inputenc}
\usepackage{indentfirst}
\usepackage{color}
\usepackage{graphics}
\usepackage{geometry}
\usepackage{amsmath,amssymb,amsfonts}
% ---- PAGE LAYOUT ----
\geometry{a4paper, top=1.5cm, bottom=1.5cm, left=1.5cm,
right=1.5cm, nohead}

\begin{document}

\begin{center}
\large{\textbf{ZESTAW ZADA� II}}
\end{center}

\noindent
\textbf{Zadanie 1} W oparciu o znane wzory i regu�y r�niczkowania oblicz pochodne podanych funkcji:\\
\textbf{(a)} $y = \sqrt{x^3}$, \textbf{(b)} $y = \sqrt[3]{x^2}$, \textbf{(c)} $y = \frac{1}{\sqrt[4]{x^3}}$, \textbf{(d)} $y = \frac{3}{x^2} - 2\sqrt{x^3} + \frac{5}{\sqrt[4]{x^3}}$,\\ 
\textbf{(e)} $y = x^2 \sin x$, \textbf{(f)} $y = \frac{1}{\ln x}$, \textbf{(g)} $y = \frac{x^2 e^x}{1-x^3}$, \textbf{(h)} $y = \sin(5x)$, \textbf{(i)} $y = \sqrt{x^2+1}$,\\ 
\textbf{(j)} $y = \text{tg}\,(x^3)$, \textbf{(k)} $y = \arcsin^3 x$, \textbf{(l)} $y = e^{x^3\cos x}$, \textbf{(m)} $y = x^3\text{arctg}^5 x$, \textbf{(n)} $y = \ln\frac{3x+2}{2x+3}$.\\

\noindent
\textbf{Zadanie 4} Oblicz dwie pierwsze pochodne podanych funkcji:\\
\textbf{(a)} $y = e^{2x}\sin(3x)$, \textbf{(b)} $y = \ln(x^2 - 3x + 1)$, 
\textbf{(c)} $y = \frac{x}{x^2 + 1}$, \textbf{(d)} $y = x\,\text{arctg}\,x$.\\





\end{document}