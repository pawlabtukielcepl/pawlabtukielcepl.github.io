\documentclass[10pt,a4paper]{article}

\usepackage{float} %dzieki temu pakietowi rysunki pojawiaja sie tam gdzie chce po uzyciu opcji H w srodowisku figure
\usepackage[font=small,labelfont=bf]{caption}
\usepackage{natbib}
\usepackage{import}
\usepackage[tight]{subfigure}
\usepackage{tikz}
\usepackage{pgfplots}
\usepackage{polski}
\usepackage[cp1250]{inputenc}
\usepackage{indentfirst}
\usepackage{color}
\usepackage{graphics}
\usepackage{geometry}
% ---- PAGE LAYOUT ----
\geometry{a4paper, top=1.5cm, bottom=1.5cm, left=1.5cm, right=1.5cm, nohead}

\usepackage{amsmath,amssymb,amsfonts}
\begin{document}
	
\clearpage
\thispagestyle{empty}

\noindent
Wskaz�wki:\\
(b) podstawi� $z = x + iy$ dla $x,y\in\mathbb{R}$, a nast�pnie $y = t\cdot x$\\
(c) punkty $z_1, z_2, z_3, z_4, z_5$ maj� posta� $z_k = x_k +(m\cdot x_k + 3)i$, $k=1,2,3,4,5$.\\


\end{document}