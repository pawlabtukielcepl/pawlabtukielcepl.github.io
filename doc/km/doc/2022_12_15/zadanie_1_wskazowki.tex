\documentclass[10pt,a4paper]{article}

\usepackage{float} %dzieki temu pakietowi rysunki pojawiaja sie tam gdzie chce po uzyciu opcji H w srodowisku figure
\usepackage[font=small,labelfont=bf]{caption}
\usepackage{natbib}
\usepackage{import}
\usepackage[tight]{subfigure}
\usepackage{tikz}
\usepackage{pgfplots}
\usepackage{polski}
\usepackage[cp1250]{inputenc}
\usepackage{indentfirst}
\usepackage{color}
\usepackage{graphics}
\usepackage{geometry}
% ---- PAGE LAYOUT ----
\geometry{a4paper, top=1.5cm, bottom=1.5cm, left=1.5cm, right=1.5cm, nohead}

\usepackage{amsmath,amssymb,amsfonts}
\begin{document}
	
\clearpage
\thispagestyle{empty}

\noindent
Wskaz�wki:\\
(a) $2222\equiv 3\ (\text{mod}\ 7)$, $3^6 \equiv 1\ (\text{mod}\ 7)$, $5555 \equiv 4\ (\text{mod}\ 7)$, $4^3 \equiv 1\ (\text{mod}\ 7)$,\\
(b) $1000 = 8\cdot 125$, $n = (4\cdot 0 + 3)\cdot (4\cdot 1 + 3)\cdot \ldots (4\cdot 505 + 3)$,\\
(c) wyznacz reszt� z dzielenia $14^{14^{14}}$ przez $25$, nast�pnie poka�, �e $14^{14^{14}}$ jest podzielna przez $4$,\\
(d) $257^{33} \equiv 7\ (\text{mod}\ 50)$, $46\equiv -4\ (\text{mod}\ 50)$.


\end{document}