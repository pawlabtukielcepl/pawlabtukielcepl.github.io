\documentclass[11pt,a4paper]{report}

\usepackage{float} %dzieki temu pakietowi rysunki pojawiaja sie tam gdzie chce po uzyciu opcji H w srodowisku figure
\usepackage[font=small,labelfont=bf]{caption}
\usepackage{natbib}
\usepackage{import}
\usepackage[tight]{subfigure}
\usepackage{tikz}
\usepackage{pgfplots}
\usepackage{polski}
\usepackage[cp1250]{inputenc}
\usepackage{indentfirst}
\usepackage{color}
\usepackage{graphics}
\usepackage{geometry}
\usepackage{amsmath,amssymb,amsfonts}
% ---- PAGE LAYOUT ----
\geometry{a4paper, top=1.5cm, bottom=1.5cm, left=1.5cm,
right=1.5cm, nohead}

\begin{document}

\begin{center}
\large{\textbf{ZESTAW ZADA� IV}}
\end{center}

\noindent
\textbf{Zadanie 1}\\
\textbf{(a)} Zapisz wz�r Taylora dla funkcji $f(x) = \ln(x+1)$ w okolicy $x_0 = 0$ z dok�adno�ci� do $n$ wyraz�w; wykorzystaj otrzymany wz�r do obliczenia przybli�enia $\ln 2$ (warto�� wskazana przez kalkulator: $0{,}693147$),\\
\textbf{(b)} Zapisz wz�r Taylora dla funkcji $f(x) = \frac{2x}{2-x}$ z dok�adno�ci� do dw�ch wyraz�w w okolicy $x_0 = 1$; wykorzystaj otrzymany wz�r do przybli�enia warto�ci funkcji dla $x = 0{,}9$,\\
\textbf{(c)} w oparciu o wz�r Taylora przybli� funkcj� $y = \sqrt{8 - x^2}$ w okolicy $x_0 = 2$ za pomoc� paraboli; sprawd� dok�adno�� przybli�enia w punktach $x = 2{,}5$ oraz $x = 2{,}1$.\\

\noindent
\textbf{Zadanie 2} \\ 
Wyznacz przedzia�y monotoniczno�ci i ekstrema lokalne podanych funkcji:\\
\textbf{(a)} $y = -2 x^3+4 x^2+8 x+10$, \textbf{(b)} $y = -3x^4 + 20x^3 - 24x^2 - 72x + 11$, \textbf{(c)} $y = 3x + \frac{1}{x^3}$,\\
\textbf{(d)} $y = x^5 + (1-x)^5$,  \textbf{(e)} $y = x^4(2x-3)^6$, \textbf{(f)} $y = \frac{x}{x^2+4}$,\\
\textbf{(g)} $y = \frac{2x^2-5x+2}{3x^2-10x+3}$, \textbf{(h)} $y = x^2\ln x$, \textbf{(i)} $y = x^3 e^{-2x}$.\\



\end{document}