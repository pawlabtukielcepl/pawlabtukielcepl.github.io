\documentclass[11pt,a4paper]{report}

\usepackage{float} %dzieki temu pakietowi rysunki pojawiaja sie tam gdzie chce po uzyciu opcji H w srodowisku figure
\usepackage[font=small,labelfont=bf]{caption}
\usepackage{natbib}
\usepackage{import}
\usepackage[tight]{subfigure}
\usepackage{tikz}
\usepackage{pgfplots}
\usepackage{polski}
\usepackage[cp1250]{inputenc}
\usepackage{indentfirst}
\usepackage{color}
\usepackage{graphics}
\usepackage{geometry}
\usepackage{amsmath,amssymb,amsfonts}
% ---- PAGE LAYOUT ----
\geometry{a4paper, top=1.5cm, bottom=1.5cm, left=1.5cm,
right=1.5cm, nohead}

\begin{document}

\begin{center}
\large{\textbf{ZESTAW ZADA� XII}}
\end{center}

\noindent
\textbf{Zadanie 1} Dane s� wektory $\vec{u} = -2\vec{j} + \vec{k} + 2\vec{i}$, $\vec{v} = 4\vec{k} - \vec{i} + 8\vec{j}$, $\vec{w} = 4\vec{j} - 7\vec{k} - 4\vec{i}$.\\
\textbf{(a)} oblicz $\vec{u}\dot\vec{v}$, \textbf{(b)} oblicz k�t pomi�dzy wektorami $\vec{u}$ i $\vec{w}$\\
\textbf{(c)} oblicz $\vec{u}\times \vec{v}$, \textbf{(d)} wyznacz pole r�wnoleg�oboku rozpi�tego przez wektory $\vec{u}$ i $\vec{w}$,\\
\textbf{(e)} wyznacz obj�to�� r�wnoleg�o�cianu i czworo�cianu rozpi�tych na wektorach $\vec{u}$, $\vec{v}$ i $\vec{w}$.\\

\noindent
\textbf{Zadanie 2} Dany jest tr�jk�t $ABC$, przy czym $A(2,-3,3)$, $B(1,-1,1)$, $C(4,-1,2)$. Wyznacz kosinus $\angle ABC$, pole tr�jk�ta $ABC$ oraz d�ugo�� wysoko�ci opuszczonej na bok $AB$.\\

\noindent
\textbf{Zadanie 3} Dane s� wektory $\vec{a} = 4\vec{j} - 4\vec{i} = 2\vec{k}$, $\vec{b} = -2\vec{i} + 2\vec{j}$, $\vec{c} = 4\vec{i} + 3\vec{k} + 2\vec{j}$ zaczepione w punkcie $A(1,-2,3)$. Rozwa�my czworo�cian $ABCD$, gdzie $\overrightarrow{AB} =  \vec{a}$,  $\overrightarrow{AC} =  \vec{b}$, $\overrightarrow{AD} =  \vec{c}$. Wyznacz wsp�rz�dne wierzcho�k�w $B$, $C$ i $D$, obj�to�� czworo�cianu, pole �ciany $ABC$, d�ugo�� wysoko�ci opuszczonej na �cian� $ABC$.\\

\end{document}