\documentclass[11pt,a4paper]{report}

\usepackage{float} %dzieki temu pakietowi rysunki pojawiaja sie tam gdzie chce po uzyciu opcji H w srodowisku figure
\usepackage[font=small,labelfont=bf]{caption}
\usepackage{natbib}
\usepackage{import}
\usepackage[tight]{subfigure}
\usepackage{tikz}
\usepackage{pgfplots}
\usepackage{polski}
\usepackage[cp1250]{inputenc}
\usepackage{indentfirst}
\usepackage{color}
\usepackage{graphics}
\usepackage{geometry}
\usepackage{amsmath,amsfonts,amssymb}
% ---- PAGE LAYOUT ----
\geometry{a4paper, top=1.5cm, bottom=1.5cm, left=1.5cm,
right=1.5cm, nohead}
\begin{document}
\begin{center}
\large{\textbf{Egzamin z matematyki 1 (WI�GiE/OZE, sesja poprawkowa), 16/02/2023}}
\end{center}

\noindent
\textbf{Zadanie 1 (0-10 pkt.)} 
Oblicz pochodne: 
$\left(\frac{1}{x} + \frac{2}{\sqrt{x}}\right)'$, 
$\left( \frac{2^x}{\text{arctg}\,x} \right)'$, 
$\left( x^4 e^{\sin x} \right)'$.
\\


\noindent
\textbf{Zadanie 2 (0-10 pkt.)}  
Wyznacz przedzia�y monotoniczno�ci i ekstrema lokalne funkcji: 
$y = -3 x^4-8 x^3+12 x^2+48 x$.
\\


\noindent
\textbf{Zadanie 3 (0-10 pkt.)}  
(a) Zapisz liczb� $z = \frac{1}{2-i} + \frac{3i}{1+2i}$ w postaci $a + bi$, gdzie $a,b$ -- liczby rzeczywiste.\\ 
(b) Rozwi�� r�wnanie $z^2 - 6z + 13 = 0$ w dziedzinie zespolonej.
\\

\noindent
\textbf{Zadanie 4 (0-10 pkt.)}  
Oblicz ca�k�: 
$\int \frac{3x-5}{x^2-4x+3} dx$.\\

\noindent
\textbf{Zadanie 5 (0-10 pkt.)}  
Wyznacz pole obszaru ograniczonego liniami  $y = 2x - x^2$, $y = 2-x$. Wykonaj rysunek!
\\

\noindent
\textbf{Zadanie 6 (0-10 pkt.)} 
Rozwi�� uk�ad r�wna� metod� Gaussa eliminacji:
$$
\left\{
\begin{array}{l}
  x + 2y + z =  -3\\
 2x + y +  2z =  6\\
2x + 2y + z = -1\\
\end{array}
\right.
$$ 

\noindent
\textbf{Zadanie 7 (0-10 pkt.)}  
W oparciu o definicj� oblicz pochodn� podanej funkcji $f(x) = 3x - x^2$ w punkcie $x_0 = 2$. Zapisz r�wnanie stycznej do wykresu funkcji w punkcie $(x_0,f(x_0))$, naszkicuj pogl�dowy wykres funkcji oraz stycznej.
\\ 

\noindent
\textbf{Zadanie 8 (0-10 pkt.)} 
Dane s� wektory $\vec{u} = 2\vec{j} - 3\vec{i} + 6\vec{k}$ oraz $\vec{v} = -6\vec{k} + 3\vec{i} + 6\vec{j}$. Wyznacz kosinus k�ta pomi�dzy nimi oraz pole tr�jk�ta rozpi�tego na tych wektorach.\\






\end{document}