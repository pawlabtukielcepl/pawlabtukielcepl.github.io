\documentclass{beamer}

\usetheme{Berlin}

\usepackage{polski}
\usepackage[cp1250]{inputenc}


\title{Egzamin z Matematyki 1 (WISGiE/OZE, sesja poprawkowa)}
\date{16/02/2021}

\begin{document}
	
\frame{\titlepage}

\begin{frame}[t]
\frametitle{Zadanie 1 (0 - 10 pkt.)}

\noindent
\Large{
Oblicz pochodne: 
$$\left(\frac{1}{x} - \frac{2}{\sqrt{x}}\right)'$$ 
$$\left( \frac{2^x}{\text{arctg}\,x} \right)'$$ 
$$\left( x^4 e^{\sin x} \right)'$$
}

\end{frame}

\begin{frame}[t]
\frametitle{Zadanie 2 (0 - 10 pkt.)}

\noindent
\Large{
Wyznacz przedzia�y monotoniczno�ci i ekstrema lokalne funkcji: 
$$y = -3 x^4-8 x^3+12 x^2+48 x$$
}

\end{frame}

\begin{frame}[t]
\frametitle{Zadanie 3 (0 - 10 pkt.)}

\noindent
\Large{
(a) Zapisz liczb� $z = \frac{1}{2-i} + \frac{3i}{1+2i}$ w postaci $a + bi$, gdzie $a,b$ -- liczby rzeczywiste.\\ 
(b) Rozwi�� r�wnanie $z^2 - 6z + 13 = 0$ w dziedzinie zespolonej.
}
\end{frame}

\begin{frame}[t]
\frametitle{Zadanie 4 (0 - 10 pkt.)}

\noindent
\Large{
Oblicz ca�k�: 
$$\int \frac{3 x-5}{x^2-4 x+3} dx$$
}

\end{frame}

\begin{frame}[t]
\frametitle{Zadanie 5 (0 - 10 pkt.)}

\noindent
\Large{
Wyznacz pole obszaru ograniczonego liniami  
$$y = 2x - x^2,\ y = 2 - x$$ 
Wykonaj rysunek!
}


\end{frame}

\begin{frame}[t]
\frametitle{Zadanie 6 (0 - 10 pkt.)}

\noindent
\Large{
Rozwi�� uk�ad r�wna� metod� Gaussa eliminacji:
$$
\left\{
\begin{array}{l}
\begin{array}{l}
	x + 2y + z =  -3\\
	2x + y +  2z =  6\\
	2x + 2y + z = -1\\
\end{array}
\end{array}
\right.
$$ 
}

\noindent
Zadanie 6


\end{frame}

\begin{frame}[t]
\frametitle{Zadanie 7 (0 - 20 pkt.)}\

\noindent
\Large{
W oparciu o definicj� oblicz pochodn� podanej funkcji $f(x) =3x - x^2$ w punkcie $x_0 = 2$. Zapisz r�wnanie stycznej do wykresu funkcji w punkcie $(x_0,f(x_0))$, naszkicuj pogl�dowy wykres funkcji oraz stycznej.
}

\end{frame}

\begin{frame}[t]
\frametitle{Zadanie 8 (0 - 20 pkt.)} 

\noindent
\Large{
Dane s� wektory $\vec{u} = 2\vec{j} - 3\vec{i} + 6\vec{k}$ oraz $\vec{v} = -6\vec{k} + 3\vec{i} + 6\vec{j}$. Wyznacz kosinus k�ta pomi�dzy nimi oraz pole tr�jk�ta rozpi�tego na tych wektorach.
}
\end{frame}


\end{document}