\documentclass[11pt,a4paper]{report}

\usepackage{float} %dzieki temu pakietowi rysunki pojawiaja sie tam gdzie chce po uzyciu opcji H w srodowisku figure
\usepackage[font=small,labelfont=bf]{caption}
\usepackage{natbib}
\usepackage{import}
\usepackage[tight]{subfigure}
\usepackage{tikz}
\usepackage{pgfplots}
\usepackage{polski}
\usepackage[cp1250]{inputenc}
\usepackage{indentfirst}
\usepackage{color}
\usepackage{graphics}
\usepackage{geometry}
\usepackage{amsmath,amssymb,amsfonts}
% ---- PAGE LAYOUT ----
\geometry{a4paper, top=1.5cm, bottom=1.5cm, left=1.5cm,
right=1.5cm, nohead}

\begin{document}

\begin{center}
\large{\textbf{ZESTAW ZADA� III}}
\end{center}

\noindent
\textbf{Zadanie 1}\\
\textbf{(a)} W oparciu o definicj� oblicz pochodn� funkcji $f(x) = 2x^2 + 3x + 4$ w punkcie $x_0 = -1$, zapisz r�wnanie stycznej do wykresu funkcji w punkcie $(x_0, f(x_0))$,\\
\textbf{(b)} W oparciu o definicj� wyprowad� wz�r na pochodn� funkcji $f(x) = \frac{x}{2x+3}$.\\


\noindent
\textbf{Zadanie 2} Oblicz pochodne:\\
\textbf{(a)} $\left(x^6 - 3x^4 +5x^3 - 6x - 5\right)'$, \textbf{(b)} $\left(\frac{1}{x^4} - \sqrt[4]{x^3} + \frac{10}{\sqrt[5]{x^3}}\right)'$, \textbf{(c)} $\left(\frac{x^3+1}{x^3-1}\right)'$, 
\textbf{(d)} $\left(3^x \text{arctg}\,x\right)'$, \\
\textbf{(e)} $\left(\frac{x^5\text{tg}\,x}{\arcsin x}\right)'$,
\textbf{(f)} $\left(\sqrt{x^2+1}\right)'$, \textbf{(g)} $\left(\sin(5x)\right)'$,  \textbf{(h)} $\left(\text{ctg}\,x^3\right)'$, 
\textbf{(i)} $\left(\text{arctg}^3\,x\right)'$, \textbf{(j)} $\left(2^{x^2\sin x}\right)'$,\\ 
\textbf{(k)} $\left(\sin^5(x^3)\right)'$, \textbf{(l)} $\left(x^3\ln\frac{3x-2}{2x+3}\right)'$, 
\textbf{(m)} $\left(x^{\frac{1}{x}}\right)'$, \textbf{(n)} $\left((x^2+1)^{\cos(3x)}\right)'$\\ 

\noindent
\textbf{Zadanie 3} Oblicz dwie pierwsze pochodne podanych funkcji:\\
\textbf{(a)} $y = e^{2x}\cos(3x)$, \textbf{(b)} $y = \ln(x^2-3x+4)$, \textbf{(c)} $y = \sin^3 x$, \textbf{(d)} $y = x\,\text{arctg}\,x^2$\\



\end{document}