\documentclass[11pt,a4paper]{report}

\usepackage{float} %dzieki temu pakietowi rysunki pojawiaja sie tam gdzie chce po uzyciu opcji H w srodowisku figure
\usepackage[font=small,labelfont=bf]{caption}
\usepackage{natbib}
\usepackage{import}
\usepackage[tight]{subfigure}
\usepackage{tikz}
\usepackage{pgfplots}
\usepackage{polski}
\usepackage[cp1250]{inputenc}
\usepackage{indentfirst}
\usepackage{color}
\usepackage{graphics}
\usepackage{geometry}
\usepackage{amsmath,amssymb,amsfonts}
% ---- PAGE LAYOUT ----
\geometry{a4paper, top=1.5cm, bottom=1.5cm, left=1.5cm,
right=1.5cm, nohead}

\begin{document}

\begin{center}
\large{\textbf{ZESTAW ZADA� VI}}
\end{center}

\noindent
\textbf{Zadanie 1} Zapisz liczby w postaci $a+bi$, gdzie $a,b$ -- liczby rzeczywiste, $i$ -- jednostka urojona zdefiniowana r�wnaniem $i^2 = -1$:\\ 
\textbf{(a)} $i + 2i^3 + 3i^6 + 4i^9 + 5i^{12}$, \textbf{(b)} $(2+3i)^2$, \textbf{(c)} $(1-i)(3+i)$, \textbf{(d)} $(1+i)^2-(3+i)^3$,\\ 
\textbf{(e)} $\frac{1}{i} + \frac{2}{i^3} + \frac{3}{i^5} + \frac{4}{i^7} + \frac{5}{i^9}$,  \textbf{(f)} $\frac{1-i}{1+i}$, 
\textbf{(g)} $\frac{4+3 i}{2-i}+(1-3 i)\cdot(-2+2 i)$, \textbf{(h)} $\frac{1}{1+2i} - \frac{2+i}{-1+3i}$.\\

\noindent
\textbf{Zadanie 2} Rozwi�� r�wnania w dziedzinie zespolonej:\\
\textbf{(a)} $z^2-4 z+13 = 0$, \textbf{(b)} $z^4+7 z^2+12 = 0$,  
\textbf{(c)} $2 z^3-3 z^2+8 z-12 = 0$, \textbf{(d)} $z^2 + 8 - 6i = 0$,\\
\textbf{(e)} $z^3 - 27 = 0$, \textbf{(f)} $z^4 + 16i = 0$, 
\textbf{(g)} $z^3 - i = 0$, \textbf{(h)} $z^2-2 i z+ 2-4 i = 0$.\\

\noindent
\textbf{Zadanie 3}\\
\textbf{(a)} Zapisz liczby w postaci trygonometrycznej $z = r(\cos\varphi + i\sin\varphi)$: $2i$, $-4$, $1 + i\sqrt{3}$, $-\sqrt{3} + i$,\\
\textbf{(b)} W oparciu o wz�r de Moivre'a:
$$
z^n = \left[r(\cos\varphi + i\sin\varphi)\right]^n = r^n \left(\cos(n\varphi) + i\cdot \sin(n\varphi)\right),\ n=1,2,\ldots
$$ 
zapisz liczby $(1+i\sqrt{3})^5$, $(-\sqrt{3} + i)^{13}$, $(-\sqrt{2} - i\sqrt{2})^{10}$ w postaci $a+bi$, gdzie $a,b$ -- liczby rzeczywiste,\\
\textbf{(c)} W oparciu o wz�r de Moivre'a rozwi�� r�wnania: $z^3 - 27 = 0$, $z^4 + 16i = 0$, $z^3 - i = 0$.\\

\end{document}