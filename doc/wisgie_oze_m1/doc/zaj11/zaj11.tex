\documentclass[11pt,a4paper]{report}

\usepackage{float} %dzieki temu pakietowi rysunki pojawiaja sie tam gdzie chce po uzyciu opcji H w srodowisku figure
\usepackage[font=small,labelfont=bf]{caption}
\usepackage{natbib}
\usepackage{import}
\usepackage[tight]{subfigure}
\usepackage{tikz}
\usepackage{pgfplots}
\usepackage{polski}
\usepackage[cp1250]{inputenc}
\usepackage{indentfirst}
\usepackage{color}
\usepackage{graphics}
\usepackage{geometry}
\usepackage{amsmath,amssymb,amsfonts}
% ---- PAGE LAYOUT ----
\geometry{a4paper, top=1.5cm, bottom=1.5cm, left=1.5cm,
right=1.5cm, nohead}

\begin{document}

\begin{center}
\large{\textbf{ZESTAW ZADA� XI}}
\end{center}

\noindent
\textbf{Zadanie 1} Oblicz:
$$
\textbf{(a)}\ 
\left |
\begin{array}{cc}
2 & -3\\
4 & -1\\
\end{array}
\right |,\ \ \ 
\textbf{(b)}\ 
\left |
\begin{array}{ccc}
-1 & 2 & 2\\
-2 & 3 & 0\\
3 & -2 & -1\\
\end{array}
\right |
$$

\noindent
\textbf{Zadanie 2} Oblicz wyznacznik z Zadania 1(b) stosuj�c rozwini�cie Laplace'a wzgl�dem:\\
\textbf{(a)} $2$--go wiersza, \textbf{(b)} $1$--ej kolumny\\

\noindent
\textbf{Zadanie 3} Oblicz:
$$
\textbf{(a)}\ 
\left|
\begin{array}{cccc}
3 & 3 & 3 & 2 \\
-4 & -4 & -3 & -3 \\
-3 & 4 & 2 & -1 \\
0 & 0 & -4 & 2 \\
\end{array}
\right|,\ \ \ 
\textbf{(b)}\ 
\left|
\begin{array}{cccc}
2 & 2 & 0 & -1 \\
0 & 3 & -1 & -3 \\
-3 & -1 & 4 & -1 \\
-2 & -2 & 0 & 1 \\
\end{array}
\right|
$$

\noindent
\textbf{Zadanie 4} Rozwi�� uk�ady r�wna� za pomoc� wzor�w Cramera:
$$
\textbf{(a)}
\left \{
\begin{array}{l}
	2 x+3 y+z = 2\\
	-2 x+y+3 z = 6\\
	3 x+2 y-2 z = -3\\
\end{array}
\right .,
\textbf{(b)}\ 
\left\{
\begin{array}{l}
	x+2 y-3 z = -5\\
	-3 x+2 y+z = -1\\
	-2 y+z = 1\\
\end{array}
\right .
$$

\noindent
\textbf{Zadanie 2} Za pomoc� wzor�w Cramera wyznacz wskazane niewiadome z uk�ad�w r�wna�:\\
$$
\textbf{(a)}\ x = ?,\ z = ?\ 
\left \{
\begin{array}{l}
	-x+2 y-3 z + t = -3\\
	3 x+3 y + t = 4\\
	x-2 y+z + t = 5\\
	2 x+y+2 z - t = 2\\
\end{array}
\right .,\ 
\textbf{(b)}\ x_2 = ?,\ x_3 = ?\ 
\left \{
\begin{array}{l}
	x_1+2 x_2-x_5 = -1\\
	2 x_1-x_3+x_5 = 0\\
	2 x_2-2 x_3+x_4 = -2\\
	x_1-2 x_2+2 x_3-x_4+x_5 = 3\\
	-x_1-x_2+2 x_4-3 x_5 = 8\\
\end{array}
\right .
$$

\end{document}