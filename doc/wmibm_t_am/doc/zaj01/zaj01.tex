\documentclass[11pt,a4paper]{report}

\usepackage{float} %dzieki temu pakietowi rysunki pojawiaja sie tam gdzie chce po uzyciu opcji H w srodowisku figure
\usepackage[font=small,labelfont=bf]{caption}
\usepackage{natbib}
\usepackage{import}
\usepackage[tight]{subfigure}
\usepackage{tikz}
\usepackage{pgfplots}
\usepackage{polski}
\usepackage[cp1250]{inputenc}
\usepackage{indentfirst}
\usepackage{color}
\usepackage{graphics}
\usepackage{geometry}
\usepackage{amsmath,amssymb,amsfonts}
% ---- PAGE LAYOUT ----
\geometry{a4paper, top=1.5cm, bottom=1.5cm, left=1.5cm,
right=1.5cm, nohead}

\begin{document}

\begin{center}
\large{\textbf{ZESTAW ZADA� I}}
\end{center}

\noindent
\textbf{Zadanie 1} Rozwi�� nier�wno�ci:\\
\textbf{(a)} $x^2-2x\geq 0$, \textbf{(b)} $9 - 4x^2 \geq 0$, \textbf{(c)} $4x^2 - 4x + 1 \leq 0$,
\textbf{(d)} $2x^2+x-3\leq 0$,\\ 
\textbf{(e)} $x^4+3 x^3+2 x^2 > 0$, \textbf{(f)} $x^3+2 x^2-x-2 \geq 0$, \textbf{(g)} $-x^4+5 x^2-4 < 0$, \textbf{(h)} $x^7-8 x^5-9 x^3 < 0$,\\ 
\textbf{(i)} $-x^3+2 x^2+5 x-6 \leq 0$, \textbf{(j)} $x^3-3 x+2 \geq 0$, \textbf{(k)} $x^4+7 x^3+8 x^2-28 x-48>0$.\\

\noindent
\textbf{Zadanie 2} Skr�� u�amki:\\
\textbf{(a)} $\frac{x^2-x}{x^3-3x^2}$, \textbf{(b)} $\frac{x^2-4}{x^2+4x+4}$, \textbf{(c)} $\frac{x^2-2 x-3}{x^2-x-6}$, 
\textbf{(d)} $\frac{x^4+x^3-x^2+x-2}{x^4+4 x^3+3 x^2-4 x-4}$.\\

\noindent
\textbf{Zadanie 3} Zapisz wyra�enia w postaci jednego nieskracalnego u�amka:\\
\textbf{(a)} $\frac{1}{x} - \frac{2}{x^2} + \frac{3}{x + 1}$, 
\textbf{(b)} $\frac{2}{x-1} - \frac{3}{2x} + \frac{4}{x^2-x}$, 
\textbf{(c)} $\frac{x^2}{x^4 - 1} + \frac{x}{x^2 + 1} - \frac{1}{2x - 2} - \frac{2}{x + 1}$.\\


\noindent
\textbf{Zadanie 4} Wyznacz wskazan� niewiadom� z r�wna�: \\ 
\textbf{(a)} $\frac{ab}{c} = d$, $a = ?$  
\textbf{(b)} $ab + c = bd + e$, $b = ?$ 
\textbf{(c)} $\frac{a}{3c+2} = b$, $c = ?$, 
\textbf{(d)} $\frac{c}{\sqrt{a+b}} = d$, $a =?$.\\
 
\noindent
\textbf{Zadanie 5} Rozwi�� nier�wno�ci:\\
\textbf{(a)} $\frac{x-1}{x^2-2x}\geq 0$, \textbf{(b)} $\frac{2}{2x+3}\leq 2$, \textbf{(c)} $\frac{2x}{x+1}\leq \frac{3x+2}{x+4}$, 
\textbf{(d)} $7-x \geq \frac{2x+1}{x-1}$.\\

\noindent
\textbf{Zadanie 6} Wyznacz dziedziny funkcji:\\
\textbf{(a)} $f(x) = \frac{x^2+1}{x^3+2 x^2-2 x-4}$, 
\textbf{(b)} $f(x) = \sqrt{6 x^3-x^2-10 x-3}$, 
\textbf{(c)} $f(x) = \frac{x + \sqrt{9-x^2}}{\sqrt{x^2-x-2}} + \frac{x}{\sqrt{x^2 + 4x + 4}}$, \\
\textbf{(d)} $f(x) = \sqrt{-6 + 2x - \frac{2}{(x+1)^2} + \frac{7}{x+1} + \frac{x+1}{x^2+1}}$.


\end{document}