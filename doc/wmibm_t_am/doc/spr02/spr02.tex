\documentclass[11pt,a4paper]{report}

\usepackage{float} %dzieki temu pakietowi rysunki pojawiaja sie tam gdzie chce po uzyciu opcji H w srodowisku figure
\usepackage[font=small,labelfont=bf]{caption}
\usepackage{natbib}
\usepackage{import}
\usepackage[tight]{subfigure}
\usepackage{tikz}
\usepackage{pgfplots}
\usepackage{polski}
\usepackage[cp1250]{inputenc}
\usepackage{indentfirst}
\usepackage{color}
\usepackage{graphics}
\usepackage{geometry}
\usepackage{amsmath,amssymb,amsfonts}
% ---- PAGE LAYOUT ----
\geometry{a4paper, top=1.5cm, bottom=1.5cm, left=0.5cm,
right=0.5cm, nohead}

\begin{document}

\noindent
\parbox{9cm}{
	\noindent
\hspace{4cm}
\textbf{I}

\noindent
\textbf{1}  Wyznacz pole obszaru ograniczonego liniami $y =3x - x^2$, $y = 3-x$.

\noindent
\textbf{2} Rozwi�� uk�ad r�wna� stosuj�c metod� Gaussa eliminacji:
$$
\left\{
\begin{array}{l}
	x+2 y+3 z = -5\\
	3 x+2 y+z = 5\\
	2 x+3 y+z = 4\\
\end{array}
\right .
$$

\noindent
\textbf{3} W oparciu o wzory Cramera wyznacz niewiadom� $z$ spe�niaj�c� uk�ad r�wna�:
$$
\left \{
\begin{array}{l}
	x + 2y + 2z + t = 2\\
	2x + y + z + 2t = 1\\
	2x + 2y + z + t = -1\\
	x + y - 2z + 2t = -8\\
\end{array}
\right.
$$
je�li wiadomo, �e wyznacznik macierzy uk�adu wynosi $W = 12$.

}
\hspace{1.2cm}
\parbox{9cm}{
	\noindent
\hspace{4cm}
\textbf{II}

\noindent
\textbf{1} (a) wyznacz dziedzin� funkcji $f(x) = \sqrt{\frac{1}{x-2}-\frac{1}{x+4}}$, 
(b) Wyznacz dziedzin� funkcji $f(x) = \frac{3-x^2}{x+3}$, zapisz r�wnania asymptot, naszkicuj wykres funkcji.

\noindent
\textbf{2} Oblicz pochodne: $\left(\frac{1}{x^7} - 2\sqrt[3]{x^2}\right)'$, $\left(\frac{\sin x}{\arcsin x}\right)'$, $\left((2x+1)^4 \cos(x^3) \right)'$.

\noindent
\textbf{3} Wyznacz przedzia�y monotoniczno�ci i ekstrema lokalne funkcji $y = -3 x^4-4 x^3+6 x^2+12 x$.

\noindent
\textbf{4} Oblicz 
$\int \left(\frac{5}{x^3} + \frac{1}{\sqrt[3]{x^2}}\right) dx$,  $\int \frac{x^4 dx}{(x^5 + 1)^{10}}$,  $\int x^3 \ln x dx$.

\noindent
\textbf{5} Wyznacz pole obszaru zawartego pomi�dzy liniami $y = x^2 - 2x + 1$ oraz $y = 3x - 5$.

\noindent
\textbf{6}  Wyznacz ekstrema lokalne funkcji $z = x^2 y - x^2 - 2y^2 + 6 xy$.\\	
}\\

\vspace{0.5cm}

\noindent
\parbox{9cm}{
	\noindent
\hspace{4cm}
\textbf{I}

\noindent
\textbf{1}  Wyznacz pole obszaru ograniczonego liniami $y =3x - x^2$, $y = 3-x$.

\noindent
\textbf{2} Rozwi�� uk�ad r�wna� stosuj�c metod� Gaussa eliminacji:
$$
\left\{
\begin{array}{l}
	x+2 y+3 z = -5\\
	3 x+2 y+z = 5\\
	2 x+3 y+z = 4\\
\end{array}
\right .
$$

\noindent
\textbf{3} W oparciu o wzory Cramera wyznacz niewiadom� $z$ spe�niaj�c� uk�ad r�wna�:
$$
\left \{
\begin{array}{l}
	x + 2y + 2z + t = 2\\
	2x + y + z + 2t = 1\\
	2x + 2y + z + t = -1\\
	x + y - 2z + 2t = -8\\
\end{array}
\right.
$$
je�li wiadomo, �e wyznacznik macierzy uk�adu wynosi $W = 12$.

}
\hspace{1.2cm}
\parbox{9cm}{
	\noindent
\hspace{4cm}
\textbf{II}

\noindent
\textbf{1} (a) wyznacz dziedzin� funkcji $f(x) = \sqrt{\frac{1}{x-2}-\frac{1}{x+4}}$, 
(b) Wyznacz dziedzin� funkcji $f(x) = \frac{3-x^2}{x+3}$, zapisz r�wnania asymptot, naszkicuj wykres funkcji.

\noindent
\textbf{2} Oblicz pochodne: $\left(\frac{1}{x^7} - 2\sqrt[3]{x^2}\right)'$, $\left(\frac{\sin x}{\arcsin x}\right)'$, $\left((2x+1)^4 \cos(x^3) \right)'$.

\noindent
\textbf{3} Wyznacz przedzia�y monotoniczno�ci i ekstrema lokalne funkcji $y = -3 x^4-4 x^3+6 x^2+12 x$.

\noindent
\textbf{4} Oblicz 
$\int \left(\frac{5}{x^3} + \frac{1}{\sqrt[3]{x^2}}\right) dx$,  $\int \frac{x^4 dx}{(x^5 + 1)^{10}}$,  $\int x^3 \ln x dx$.

\noindent
\textbf{5} Wyznacz pole obszaru zawartego pomi�dzy liniami $y = x^2 - 2x + 1$ oraz $y = 3x - 5$.

\noindent
\textbf{6}  Wyznacz ekstrema lokalne funkcji $z = x^2 y - x^2 - 2y^2 + 6 xy$.\\	
}\\

\vspace{0.5cm}


\noindent
\parbox{9cm}{
	\noindent
\hspace{4cm}
\textbf{I}

\noindent
\textbf{1}  Wyznacz pole obszaru ograniczonego liniami $y =3x - x^2$, $y = 3-x$.

\noindent
\textbf{2} Rozwi�� uk�ad r�wna� stosuj�c metod� Gaussa eliminacji:
$$
\left\{
\begin{array}{l}
	x+2 y+3 z = -5\\
	3 x+2 y+z = 5\\
	2 x+3 y+z = 4\\
\end{array}
\right .
$$

\noindent
\textbf{3} W oparciu o wzory Cramera wyznacz niewiadom� $z$ spe�niaj�c� uk�ad r�wna�:
$$
\left \{
\begin{array}{l}
	x + 2y + 2z + t = 2\\
	2x + y + z + 2t = 1\\
	2x + 2y + z + t = -1\\
	x + y - 2z + 2t = -8\\
\end{array}
\right.
$$
je�li wiadomo, �e wyznacznik macierzy uk�adu wynosi $W = 12$.

}
\hspace{1.2cm}
\parbox{9cm}{
	\noindent
\hspace{4cm}
\textbf{II}

\noindent
\textbf{1} (a) wyznacz dziedzin� funkcji $f(x) = \sqrt{\frac{1}{x-2}-\frac{1}{x+4}}$, 
(b) Wyznacz dziedzin� funkcji $f(x) = \frac{3-x^2}{x+3}$, zapisz r�wnania asymptot, naszkicuj wykres funkcji.

\noindent
\textbf{2} Oblicz pochodne: $\left(\frac{1}{x^7} - 2\sqrt[3]{x^2}\right)'$, $\left(\frac{\sin x}{\arcsin x}\right)'$, $\left((2x+1)^4 \cos(x^3) \right)'$.

\noindent
\textbf{3} Wyznacz przedzia�y monotoniczno�ci i ekstrema lokalne funkcji $y = -3 x^4-4 x^3+6 x^2+12 x$.

\noindent
\textbf{4} Oblicz 
$\int \left(\frac{5}{x^3} + \frac{1}{\sqrt[3]{x^2}}\right) dx$,  $\int \frac{x^4 dx}{(x^5 + 1)^{10}}$,  $\int x^3 \ln x dx$.

\noindent
\textbf{5} Wyznacz pole obszaru zawartego pomi�dzy liniami $y = x^2 - 2x + 1$ oraz $y = 3x - 5$.

\noindent
\textbf{6}  Wyznacz ekstrema lokalne funkcji $z = x^2 y - x^2 - 2y^2 + 6 xy$.\\	
}\\

\vspace{0.5cm}


\noindent
\parbox{9cm}{
	\noindent
\hspace{4cm}
\textbf{I}

\noindent
\textbf{1}  Wyznacz pole obszaru ograniczonego liniami $y =3x - x^2$, $y = 3-x$.

\noindent
\textbf{2} Rozwi�� uk�ad r�wna� stosuj�c metod� Gaussa eliminacji:
$$
\left\{
\begin{array}{l}
	x+2 y+3 z = -5\\
	3 x+2 y+z = 5\\
	2 x+3 y+z = 4\\
\end{array}
\right .
$$

\noindent
\textbf{3} W oparciu o wzory Cramera wyznacz niewiadom� $z$ spe�niaj�c� uk�ad r�wna�:
$$
\left \{
\begin{array}{l}
	x + 2y + 2z + t = 2\\
	2x + y + z + 2t = 1\\
	2x + 2y + z + t = -1\\
	x + y - 2z + 2t = -8\\
\end{array}
\right.
$$
je�li wiadomo, �e wyznacznik macierzy uk�adu wynosi $W = 12$.

}
\hspace{1.2cm}
\parbox{9cm}{
	\noindent
\hspace{4cm}
\textbf{II}

\noindent
\textbf{1} (a) wyznacz dziedzin� funkcji $f(x) = \sqrt{\frac{1}{x-2}-\frac{1}{x+4}}$, 
(b) Wyznacz dziedzin� funkcji $f(x) = \frac{3-x^2}{x+3}$, zapisz r�wnania asymptot, naszkicuj wykres funkcji.

\noindent
\textbf{2} Oblicz pochodne: $\left(\frac{1}{x^7} - 2\sqrt[3]{x^2}\right)'$, $\left(\frac{\sin x}{\arcsin x}\right)'$, $\left((2x+1)^4 \cos(x^3) \right)'$.

\noindent
\textbf{3} Wyznacz przedzia�y monotoniczno�ci i ekstrema lokalne funkcji $y = -3 x^4-4 x^3+6 x^2+12 x$.

\noindent
\textbf{4} Oblicz 
$\int \left(\frac{5}{x^3} + \frac{1}{\sqrt[3]{x^2}}\right) dx$,  $\int \frac{x^4 dx}{(x^5 + 1)^{10}}$,  $\int x^3 \ln x dx$.

\noindent
\textbf{5} Wyznacz pole obszaru zawartego pomi�dzy liniami $y = x^2 - 2x + 1$ oraz $y = 3x - 5$.

\noindent
\textbf{6}  Wyznacz ekstrema lokalne funkcji $z = x^2 y - x^2 - 2y^2 + 6 xy$.\\	
}\\

\vspace{0.5cm}


\noindent
\parbox{9cm}{
	\noindent
\hspace{4cm}
\textbf{I}

\noindent
\textbf{1}  Wyznacz pole obszaru ograniczonego liniami $y =3x - x^2$, $y = 3-x$.

\noindent
\textbf{2} Rozwi�� uk�ad r�wna� stosuj�c metod� Gaussa eliminacji:
$$
\left\{
\begin{array}{l}
	x+2 y+3 z = -5\\
	3 x+2 y+z = 5\\
	2 x+3 y+z = 4\\
\end{array}
\right .
$$

\noindent
\textbf{3} W oparciu o wzory Cramera wyznacz niewiadom� $z$ spe�niaj�c� uk�ad r�wna�:
$$
\left \{
\begin{array}{l}
	x + 2y + 2z + t = 2\\
	2x + y + z + 2t = 1\\
	2x + 2y + z + t = -1\\
	x + y - 2z + 2t = -8\\
\end{array}
\right.
$$
je�li wiadomo, �e wyznacznik macierzy uk�adu wynosi $W = 12$.

}
\hspace{1.2cm}
\parbox{9cm}{
	\noindent
\hspace{4cm}
\textbf{II}

\noindent
\textbf{1} (a) wyznacz dziedzin� funkcji $f(x) = \sqrt{\frac{1}{x-2}-\frac{1}{x+4}}$, 
(b) Wyznacz dziedzin� funkcji $f(x) = \frac{3-x^2}{x+3}$, zapisz r�wnania asymptot, naszkicuj wykres funkcji.

\noindent
\textbf{2} Oblicz pochodne: $\left(\frac{1}{x^7} - 2\sqrt[3]{x^2}\right)'$, $\left(\frac{\sin x}{\arcsin x}\right)'$, $\left((2x+1)^4 \cos(x^3) \right)'$.

\noindent
\textbf{3} Wyznacz przedzia�y monotoniczno�ci i ekstrema lokalne funkcji $y = -3 x^4-4 x^3+6 x^2+12 x$.

\noindent
\textbf{4} Oblicz 
$\int \left(\frac{5}{x^3} + \frac{1}{\sqrt[3]{x^2}}\right) dx$,  $\int \frac{x^4 dx}{(x^5 + 1)^{10}}$,  $\int x^3 \ln x dx$.

\noindent
\textbf{5} Wyznacz pole obszaru zawartego pomi�dzy liniami $y = x^2 - 2x + 1$ oraz $y = 3x - 5$.

\noindent
\textbf{6}  Wyznacz ekstrema lokalne funkcji $z = x^2 y - x^2 - 2y^2 + 6 xy$.\\	
}\\


\end{document}