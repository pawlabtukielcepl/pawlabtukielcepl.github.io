\documentclass[11pt,a4paper]{report}

\usepackage{float} %dzieki temu pakietowi rysunki pojawiaja sie tam gdzie chce po uzyciu opcji H w srodowisku figure
\usepackage[font=small,labelfont=bf]{caption}
\usepackage{natbib}
\usepackage{import}
\usepackage[tight]{subfigure}
\usepackage{tikz}
\usepackage{pgfplots}
\usepackage{polski}
\usepackage[cp1250]{inputenc}
\usepackage{indentfirst}
\usepackage{color}
\usepackage{graphics}
\usepackage{geometry}
\usepackage{amsmath,amssymb,amsfonts}
% ---- PAGE LAYOUT ----
\geometry{a4paper, top=1.5cm, bottom=1.5cm, left=0.5cm,
right=0.5cm, nohead}

\begin{document}

\noindent
\parbox{9cm}{
	\noindent
\hspace{4cm}
\textbf{I}

\noindent
\textbf{1} Oblicz ca�k� $\int \frac{4 x+7}{x^2+x-6} dx$.

\noindent
\textbf{2} Wyznacz pole obszaru zawartego pomi�dzy liniami $y = x^2$ oraz $y = 3x$.

\noindent
\textbf{3}  Wyznacz ekstrema lokalne funkcji $z = x^2 - 2xy + y^3$.\\ 
.

}
\hspace{1.2cm}
\parbox{9cm}{
	\noindent
\hspace{4cm}
\textbf{II}

\noindent
\textbf{1} Oblicz $\int \frac{3 x^2-2 x-1}{(x+1) \left(x^2+1\right)} dx$.

\noindent
\textbf{2} Wyznacz obj�to�� bry�y powsta�ej przez obr�t linii $y = 2x - 1$ wok� osi Ox, gdy $1\leq x \leq 2$.

\noindent
\textbf{3} Wyznacz ekstrema lokalne funkcji $z = -x^2 +xy^2 - y^3 - 4x$.	
}\\

\vspace{0.5cm}

\noindent
\parbox{9cm}{
	\noindent
\hspace{4cm}
\textbf{I}

\noindent
\textbf{1} Oblicz ca�k� $\int \frac{4 x+7}{x^2+x-6} dx$.

\noindent
\textbf{2} Wyznacz pole obszaru zawartego pomi�dzy liniami $y = x^2$ oraz $y = 3x$.

\noindent
\textbf{3}  Wyznacz ekstrema lokalne funkcji $z = x^2 - 2xy + y^3$.\\ 
.

}
\hspace{1.2cm}
\parbox{9cm}{
	\noindent
\hspace{4cm}
\textbf{II}

\noindent
\textbf{1} Oblicz $\int \frac{3 x^2-2 x-1}{(x+1) \left(x^2+1\right)} dx$.

\noindent
\textbf{2} Wyznacz obj�to�� bry�y powsta�ej przez obr�t linii $y = 2x - 1$ wok� osi Ox, gdy $1\leq x \leq 2$.

\noindent
\textbf{3} Wyznacz ekstrema lokalne funkcji $z = -x^2 +xy^2 - y^3 - 4x$.	
}\\

\vspace{0.5cm}


\noindent
\parbox{9cm}{
	\noindent
\hspace{4cm}
\textbf{I}

\noindent
\textbf{1} Oblicz ca�k� $\int \frac{4 x+7}{x^2+x-6} dx$.

\noindent
\textbf{2} Wyznacz pole obszaru zawartego pomi�dzy liniami $y = x^2$ oraz $y = 3x$.

\noindent
\textbf{3}  Wyznacz ekstrema lokalne funkcji $z = x^2 - 2xy + y^3$.\\ 
.

}
\hspace{1.2cm}
\parbox{9cm}{
	\noindent
\hspace{4cm}
\textbf{II}

\noindent
\textbf{1} Oblicz $\int \frac{3 x^2-2 x-1}{(x+1) \left(x^2+1\right)} dx$.

\noindent
\textbf{2} Wyznacz obj�to�� bry�y powsta�ej przez obr�t linii $y = 2x - 1$ wok� osi Ox, gdy $1\leq x \leq 2$.

\noindent
\textbf{3} Wyznacz ekstrema lokalne funkcji $z = -x^2 +xy^2 - y^3 - 4x$.	
}\\

\vspace{0.5cm}


\noindent
\parbox{9cm}{
	\noindent
\hspace{4cm}
\textbf{I}

\noindent
\textbf{1} Oblicz ca�k� $\int \frac{4 x+7}{x^2+x-6} dx$.

\noindent
\textbf{2} Wyznacz pole obszaru zawartego pomi�dzy liniami $y = x^2$ oraz $y = 3x$.

\noindent
\textbf{3}  Wyznacz ekstrema lokalne funkcji $z = x^2 - 2xy + y^3$.\\ 
.

}
\hspace{1.2cm}
\parbox{9cm}{
	\noindent
\hspace{4cm}
\textbf{II}

\noindent
\textbf{1} Oblicz $\int \frac{3 x^2-2 x-1}{(x+1) \left(x^2+1\right)} dx$.

\noindent
\textbf{2} Wyznacz obj�to�� bry�y powsta�ej przez obr�t linii $y = 2x - 1$ wok� osi Ox, gdy $1\leq x \leq 2$.

\noindent
\textbf{3} Wyznacz ekstrema lokalne funkcji $z = -x^2 +xy^2 - y^3 - 4x$.	
}\\

\vspace{0.5cm}


\noindent
\parbox{9cm}{
	\noindent
\hspace{4cm}
\textbf{I}

\noindent
\textbf{1} Oblicz ca�k� $\int \frac{4 x+7}{x^2+x-6} dx$.

\noindent
\textbf{2} Wyznacz pole obszaru zawartego pomi�dzy liniami $y = x^2$ oraz $y = 3x$.

\noindent
\textbf{3}  Wyznacz ekstrema lokalne funkcji $z = x^2 - 2xy + y^3$.\\ 
.

}
\hspace{1.2cm}
\parbox{9cm}{
	\noindent
\hspace{4cm}
\textbf{II}

\noindent
\textbf{1} Oblicz $\int \frac{3 x^2-2 x-1}{(x+1) \left(x^2+1\right)} dx$.

\noindent
\textbf{2} Wyznacz obj�to�� bry�y powsta�ej przez obr�t linii $y = 2x - 1$ wok� osi Ox, gdy $1\leq x \leq 2$.

\noindent
\textbf{3} Wyznacz ekstrema lokalne funkcji $z = -x^2 +xy^2 - y^3 - 4x$.	
}\\


\end{document}