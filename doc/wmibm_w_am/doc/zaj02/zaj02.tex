\documentclass[11pt,a4paper]{report}

\usepackage{float} %dzieki temu pakietowi rysunki pojawiaja sie tam gdzie chce po uzyciu opcji H w srodowisku figure
\usepackage[font=small,labelfont=bf]{caption}
\usepackage{natbib}
\usepackage{import}
\usepackage[tight]{subfigure}
\usepackage{tikz}
\usepackage{pgfplots}
\usepackage{polski}
\usepackage[cp1250]{inputenc}
\usepackage{indentfirst}
\usepackage{color}
\usepackage{graphics}
\usepackage{geometry}
\usepackage{amsmath,amssymb,amsfonts}
% ---- PAGE LAYOUT ----
\geometry{a4paper, top=1.5cm, bottom=1.5cm, left=1.5cm,
right=1.5cm, nohead}

\begin{document}

\begin{center}
\large{\textbf{ZESTAW ZADA� II}}
\end{center}


\noindent
\textbf{Zadanie 1} Oblicz:\\
\textbf{(a)} stosuj�c definicj� funkcji trygonometrycznych dowolnego k�ta oblicz (o ile to mo�liwe) warto�ci funkcji sinus, kosinus, tangens i kotangens dla k�t�w: $0^{\circ}$, $90^{\circ}$, $180^{\circ}$, $270^{\circ}$, $135^{\circ}$ i $225^{\circ}$,\\
\textbf{(b)} stosuj�c wzory redukcyjne oblicz warto�ci funkcji trygonometrycznych sinus i kosinus dla k�t�w: $120^{\circ}$, $240^{\circ}$, $\frac{5\pi}{4}$, $\frac{5\pi}{3}$.\\

\noindent
\textbf{Zadanie 2} Oblicz:\\
\textbf{(a)} $\arcsin \frac{1}{2}$, \textbf{(b)} $\arccos\left(-\frac{\sqrt{3}}{2}\right)$, \textbf{(c)} $\text{arctg}\, 1$, \textbf{(d)} $\text{arcctg}\, (-\sqrt{3})$.\\

\noindent
\textbf{Zadanie 3}\\
\textbf{(a)} Zapisz wyra�enia w postaci $2^{\alpha}$, gdzie $\alpha$ -- pewna liczba wymierna: $\sqrt{2\sqrt[3]{4}}$, $\frac{\sqrt[3]{2}\sqrt[4]{8}}{(\sqrt[5]{16})^3}$.\\
\textbf{(b)} Rozwi�� r�wnania i nier�wno�ci $2^{3 x} -7\cdot 2^{2 x} + 7\cdot 2^{x+1} - 8 = 0$, $\left(\frac{1}{2}\right)^{x^2} - \frac{1}{16} \leq 0$, $2^{2x} - 5\cdot 2^x \leq 4$.\\

\noindent
\textbf{Zadanie 4}\\
\textbf{(a)} Oblicz: $\log_2 \sqrt[3]{32}$, $\log \sqrt{0{,}0001}$, $\ln \frac{1}{\sqrt[5]{e^3}}$, $\log_{\sqrt{2}} \frac{1}{16}$.\\
\textbf{(b)} Oblicz $\log 0{,}2 - \log 2$, $\log_2 \sqrt[3]{6} - \frac 13\log_2 3$, $16^{1-\log_4 3} + 2\cdot 5^{-\log_5 9}$, $\log_9 5\cdot \log_{25}27$.\\
\textbf{(c)} Rozwi�� r�wnania i nier�wno�ci: $\log_2\left(\log_3\left(\log_4 x\right)\right) = 0$, $\log_{\frac 12} (2x - 5) < -4$, $\frac{\log_2 (35 - x^3)}{\log_2 (5-x)} < 3$.\\

\noindent
\textbf{Zadanie 5} Wyznacz dziedziny funkcji:\\
\textbf{(a)} $f(x) = \ln(x^3 - 2x^2 - 5x + 6)$
\textbf{(b)} $f(x) = \frac{x - 5}{2\ln x - \ln(x + 2)}$.

\end{document}