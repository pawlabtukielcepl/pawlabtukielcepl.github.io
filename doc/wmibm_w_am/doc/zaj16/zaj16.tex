\documentclass[11pt,a4paper]{report}

\usepackage{float} %dzieki temu pakietowi rysunki pojawiaja sie tam gdzie chce po uzyciu opcji H w srodowisku figure
\usepackage[font=small,labelfont=bf]{caption}
\usepackage{natbib}
\usepackage{import}
\usepackage[tight]{subfigure}
\usepackage{tikz}
\usepackage{pgfplots}
\usepackage{polski}
\usepackage[cp1250]{inputenc}
\usepackage{indentfirst}
\usepackage{color}
\usepackage{graphics}
\usepackage{geometry}
\usepackage{amsmath,amssymb,amsfonts}
% ---- PAGE LAYOUT ----
\geometry{a4paper, top=1.5cm, bottom=1.5cm, left=1.5cm,
right=1.5cm, nohead}

\begin{document}

\begin{center}
\large{\textbf{ZESTAW ZADA� XVI}}
\end{center}

\noindent
\textbf{Zadanie 1}\\
\textbf{(a)} Oblicz pochodne $z'_x$ oraz $z'_y$, gdy: $z = y^3 \ln(x^2y^7 + 5x + 3y^4)$,\\
\textbf{(b)} Oblicz pochodn� kierunkow� funkcji $z = y^3 e^{xy}$ w punkcie $P(0,1)$ w kierunku najszybszego wzrostu funkcji.\\

\noindent
\textbf{Zadanie 2} Oblicz pochodne cz�stkowe $z''_{xx}$, $z''_{yx}$, $z''_{yy}$:\\
\textbf{(a)} $z = x^3 + 4x^2 y - 3 x y^2 - 3 y^3 + 5 x^2 - 3x y + 4y$,  \textbf{(b)} $z = \ln (x^2 - y^2)$, \textbf{(c)} $z = \frac{2x+3y}{x + y}$, 
\textbf{(d)} $z = \arcsin\frac{x}{y}$\\

\noindent
\textbf{Zadanie 3} Wyznacz ekstrema lokalne podanych funkcji:\\
\textbf{(a)} $z = (x-3y+2)^2 +(y-x-2)^2$, \textbf{(b)} $z = x^3 + x y -y^2 - x$, 
\textbf{(c)} $z = x y (3-x-y)$, \textbf{(d)} $z = x^2 y - y^2 + 4 x y$.\\



\end{document}