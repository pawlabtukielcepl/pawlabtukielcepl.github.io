\documentclass[11pt,a4paper]{report}

\usepackage{float} %dzieki temu pakietowi rysunki pojawiaja sie tam gdzie chce po uzyciu opcji H w srodowisku figure
\usepackage[font=small,labelfont=bf]{caption}
\usepackage{natbib}
\usepackage{import}
\usepackage[tight]{subfigure}
\usepackage{tikz}
\usepackage{pgfplots}
\usepackage{polski}
\usepackage[cp1250]{inputenc}
\usepackage{indentfirst}
\usepackage{color}
\usepackage{graphics}
\usepackage{geometry}
\usepackage{amsmath,amssymb,amsfonts}
% ---- PAGE LAYOUT ----
\geometry{a4paper, top=1.5cm, bottom=1.5cm, left=1.5cm,
right=1.5cm, nohead}

\begin{document}

\begin{center}
\large{\textbf{ZESTAW ZADA� IV}}
\end{center}

\noindent
\textbf{Zadanie 1} Sprawd�, �e podane funkcje spe�niaj� podane r�wnania r�niczkowe:\\
\textbf{(a)} $y = e^{-3x}$, r�wnanie: $y' + 3y = 0$, \textbf{(b)} $y = 3\cos(5x) + 5\sin(5x)$, r�wnanie $y'' + 25y = 0$, \\
\textbf{(c)} $y = 3e^{-x} + 5xe^{-x}$, r�wnanie: $y''+2y'+y = 0$,\\ 
\textbf{(d)} $y = e^{-2x}(3\cos(3x) + 2\sin(3x))$, r�wnanie: $y'' + 4y' + 13y = 0$.\\

\noindent
\textbf{Zadanie 2}\\
\textbf{(a)} Zapisz wz�r Taylora dla funkcji $f(x) = \ln(x+1)$ w okolicy $x_0 = 0$ z dok�adno�ci� do $n$ wyraz�w; wykorzystaj otrzymany wz�r do obliczenia przybli�enia $\ln 2$ (warto�� wskazana przez kalkulator: $0{,}693147$),\\
\textbf{(b)} Zapisz wz�r Taylora dla funkcji $f(x) = \frac{2x}{2-x}$ z dok�adno�ci� do dw�ch wyraz�w w okolicy $x_0 = 1$; wykorzystaj otrzymany wz�r do przybli�enia warto�ci funkcji dla $x = 0{,}9$,\\
\textbf{(c)} w oparciu o wz�r Taylora przybli� funkcj� $y = \sqrt{8 - x^2}$ w okolicy $x_0 = 2$ za pomoc� paraboli; sprawd� dok�adno�� przybli�enia w punktach $x = 2{,}5$ oraz $x = 2{,}1$.\\

\noindent
\textbf{Zadanie 3} W oparciu o regu�� de l'Hospitala oblicz poni�sze granice:\\
\textbf{(a)} $\lim\limits_{x\rightarrow 2} \frac{x^3+5 x^2-2 x-24}{x^3-2 x^2-3 x+6}$, 
\textbf{(b)} $\lim\limits_{x\rightarrow 0} \frac{x\sin x}{e^{-2x} - 1 + 2x}$, 
\textbf{(c)} $\lim\limits_{x\rightarrow 0}\frac{\sin(x^2)}{\ln(\cos x)}$, 
\textbf{(d)} $\lim\limits_{x\rightarrow\infty}\frac{e^x}{x^2}$.\\



\end{document}