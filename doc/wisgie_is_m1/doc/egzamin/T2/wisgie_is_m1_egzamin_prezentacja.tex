\documentclass{beamer}

\usetheme{Berlin}

\usepackage{polski}
\usepackage[cp1250]{inputenc}


\title{Egzamin z Matematyki 1 (WISGiE/I�, sesja poprawkowa)}
\date{16/02/2023}

\begin{document}
	
\frame{\titlepage}

\begin{frame}[t]
\frametitle{Zadanie 1 (0 - 10 pkt.)}

\noindent
\Large{
Oblicz pochodne: 
$$\left(\frac{1}{x^2} - \sqrt[3]{x^2}\right)'$$ 
$$\left( \frac{\ln x}{\text{tg}\,x} \right)'$$
$$\left( \ln\frac{x^2}{x - 1} \right)'$$
}

\end{frame}

\begin{frame}[t]
\frametitle{Zadanie 2 (0 - 10 pkt.)}

\noindent
\Large{
Zapisz wz�r Taylora dla funkcji 
$$f(x) = \sqrt[3]{x}$$ 
w okolicy $x_0 = 1$ z dok�adno�ci� do wyraz�w drugiego rz�du. Wykorzystaj uzyskany wz�r do wyznaczenia przybli�onej warto�ci $\sqrt[3]{0{,}9}$.
}

\end{frame}

\begin{frame}[t]
\frametitle{Zadanie 3 (0 - 10 pkt.)}

\noindent
\Large{
Wyznacz przedzia�y monotoniczno�ci i ekstrema lokalne funkcji: 
$$y = 3 x^4+4 x^3-30 x^2-60 x$$
}
\end{frame}

\begin{frame}[t]
\frametitle{Zadanie 4 (0 - 10 pkt.)}

\noindent
\Large{
Oblicz ca�k�: 
$$
\int \frac{x+5}{x^2+x-2} dx
$$
}

\end{frame}

\begin{frame}[t]
\frametitle{Zadanie 5 (0 - 10 pkt.)}


\Large{
\noindent
Oblicz ca�ki oznaczone: 
$$\int\limits_{1}^{4} \left( 2x - \frac{1}{\sqrt{x}} \right) dx$$ 
$$\int\limits_{0}^{2} \frac{x^3 dx}{\sqrt{x^4 + 9}}$$
}


\end{frame}

\begin{frame}[t]
\frametitle{Zadanie 6 (0 - 10 pkt.)}

\noindent
\Large{
Wyznacz pole obszaru ograniczonego liniami  
$$y = 3x - x^2,\  y = 3 - x$$ 
Wykonaj rysunek!
}

\end{frame}

\begin{frame}[t]
\frametitle{Zadanie 7 (0 - 10 pkt.)}\


\noindent
\Large{
W oparciu o definicj� oblicz pochodn� podanej funkcji 
$$f(x) = 3x - x^2$$ 
w punkcie $x_0 = 2$. Zapisz r�wnanie stycznej do wykresu funkcji w punkcie $(x_0,f(x_0))$, naszkicuj pogl�dowy wykres funkcji oraz stycznej.
}

\end{frame}

\begin{frame}[t]
\frametitle{Zadanie 8 (0 - 10 pkt.)}

\noindent
\Large{
W oparciu o rachunek ca�kowy wyznacz po�o�enie �rodka ci�ko�ci obszaru ograniczonego liniami $y = -x$, $y = 3x$, $x = 1$.
}
\end{frame}


\end{document}