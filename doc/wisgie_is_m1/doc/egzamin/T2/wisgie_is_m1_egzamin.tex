\documentclass[11pt,a4paper]{report}

\usepackage{float} %dzieki temu pakietowi rysunki pojawiaja sie tam gdzie chce po uzyciu opcji H w srodowisku figure
\usepackage[font=small,labelfont=bf]{caption}
\usepackage{natbib}
\usepackage{import}
\usepackage[tight]{subfigure}
\usepackage{tikz}
\usepackage{pgfplots}
\usepackage{polski}
\usepackage[cp1250]{inputenc}
\usepackage{indentfirst}
\usepackage{color}
\usepackage{graphics}
\usepackage{geometry}
\usepackage{amsmath,amsfonts,amssymb}
% ---- PAGE LAYOUT ----
\geometry{a4paper, top=1.5cm, bottom=1.5cm, left=1.5cm,
right=1.5cm, nohead}
\begin{document}
\begin{center}
\large{\textbf{Egzamin z matematyki 1 (WI�GiE/I�, termin pierwszy), 10/02/2023}}
\end{center}

\noindent
\textbf{Zadanie 1 (0-10 pkt.)} 
Oblicz pochodne: 
$\left(\frac{1}{x^2} - \sqrt[3]{x^2}\right)'$, 
$\left( \frac{\ln x}{\text{tg}\,x} \right)'$, 
$\left( \ln\frac{x^2}{x - 1} \right)'$.
\\


\noindent
\textbf{Zadanie 2 (0-10 pkt.)}  
Zapisz wz�r Taylora dla funkcji $f(x) = \sqrt[3]{x}$ w okolicy $x_0 = 1$ z dok�adno�ci� do wyraz�w drugiego rz�du. Wykorzystaj uzyskany wz�r do wyznaczenia przybli�onej warto�ci $\sqrt[3]{0{,}9}$.
\\


\noindent
\textbf{Zadanie 3 (0-10 pkt.)}  Wyznacz przedzia�y monotoniczno�ci i ekstrema lokalne funkcji: 
$y = 3 x^4+4 x^3-30 x^2-60 x$.
\\

\noindent
\textbf{Zadanie 4 (0-10 pkt.)}  Oblicz ca�k�: 
$\int \frac{x+5}{x^2+x-2} dx$.
\\

\noindent
\textbf{Zadanie 5 (0-10 pkt.)}  Oblicz ca�ki oznaczone: 
$\int\limits_{1}^{4} \left( 2x - \frac{1}{\sqrt{x}} \right) dx$, 
$\int\limits_{0}^{2} \frac{x^3 dx}{\sqrt{x^4 + 9}}$.
\\

\noindent
\textbf{Zadanie 6 (0-10 pkt.)}  Wyznacz pole obszaru ograniczonego liniami  $y = 3x - x^2$, $y = 3 - x$. Wykonaj rysunek!
\\ 

\noindent
\textbf{Zadanie 7 (0-20 pkt.)}  W oparciu o definicj� oblicz pochodn� podanej funkcji $f(x) = 3x - x^2$ w punkcie $x_0 = 2$. Zapisz r�wnanie stycznej do wykresu funkcji w punkcie $(x_0,f(x_0))$, naszkicuj pogl�dowy wykres funkcji oraz stycznej.\\ 

\noindent
\textbf{Zadanie 8 (0-20 pkt.)}\\ 
W oparciu o rachunek ca�kowy wyznacz po�o�enie �rodka ci�ko�ci obszaru ograniczonego liniami $y = -x$, $y = 3x$, $x = 1$.






\end{document}