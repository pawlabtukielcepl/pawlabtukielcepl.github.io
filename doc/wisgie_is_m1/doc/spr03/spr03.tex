\documentclass[11pt,a4paper]{report}

\usepackage{float} %dzieki temu pakietowi rysunki pojawiaja sie tam gdzie chce po uzyciu opcji H w srodowisku figure
\usepackage[font=small,labelfont=bf]{caption}
\usepackage{natbib}
\usepackage{import}
\usepackage[tight]{subfigure}
\usepackage{tikz}
\usepackage{pgfplots}
\usepackage{polski}
\usepackage[cp1250]{inputenc}
\usepackage{indentfirst}
\usepackage{color}
\usepackage{graphics}
\usepackage{geometry}
\usepackage{amsmath,amssymb,amsfonts}
% ---- PAGE LAYOUT ----
\geometry{a4paper, top=1.5cm, bottom=1.5cm, left=0.5cm,
right=0.5cm, nohead}

\begin{document}

\noindent
\parbox{9cm}{
	\noindent
\hspace{4cm}
\textbf{I}

\noindent
\textbf{1}  Wyznacz pole obszaru ograniczonego liniami $y =3x - x^2$, $y = 3-x$.

\noindent
\textbf{2} Rozwi�� uk�ad r�wna� stosuj�c metod� Gaussa eliminacji:
$$
\left\{
\begin{array}{l}
	x+2 y+3 z = -5\\
	3 x+2 y+z = 5\\
	2 x+3 y+z = 4\\
\end{array}
\right .
$$

\noindent
\textbf{3} W oparciu o wzory Cramera wyznacz niewiadom� $z$ spe�niaj�c� uk�ad r�wna�:
$$
\left \{
\begin{array}{l}
	x + 2y + 2z + t = 2\\
	2x + y + z + 2t = 1\\
	2x + 2y + z + t = -1\\
	x + y - 2z + 2t = -8\\
\end{array}
\right.
$$
je�li wiadomo, �e wyznacznik macierzy uk�adu wynosi $W = 12$.

}
\hspace{1.2cm}
\parbox{9cm}{
	\noindent
\hspace{4cm}
\textbf{I}

\noindent
\textbf{1}  Wyznacz pole obszaru ograniczonego liniami $y =3x - x^2$, $y = 3-x$.

\noindent
\textbf{2} Rozwi�� uk�ad r�wna� stosuj�c metod� Gaussa eliminacji:
$$
\left\{
\begin{array}{l}
	x+2 y+3 z = -5\\
	3 x+2 y+z = 5\\
	2 x+3 y+z = 4\\
\end{array}
\right .
$$

\noindent
\textbf{3} W oparciu o wzory Cramera wyznacz niewiadom� $z$ spe�niaj�c� uk�ad r�wna�:
$$
\left \{
\begin{array}{l}
	x + 2y + 2z + t = 2\\
	2x + y + z + 2t = 1\\
	2x + 2y + z + t = -1\\
	x + y - 2z + 2t = -8\\
\end{array}
\right.
$$
je�li wiadomo, �e wyznacznik macierzy uk�adu wynosi $W = 12$.
	
}\\

\vspace{0.5cm}

\noindent
\parbox{9cm}{
	\noindent
\hspace{4cm}
\textbf{I}

\noindent
\textbf{1}  Wyznacz pole obszaru ograniczonego liniami $y =3x - x^2$, $y = 3-x$.

\noindent
\textbf{2} Rozwi�� uk�ad r�wna� stosuj�c metod� Gaussa eliminacji:
$$
\left\{
\begin{array}{l}
	x+2 y+3 z = -5\\
	3 x+2 y+z = 5\\
	2 x+3 y+z = 4\\
\end{array}
\right .
$$

\noindent
\textbf{3} W oparciu o wzory Cramera wyznacz niewiadom� $z$ spe�niaj�c� uk�ad r�wna�:
$$
\left \{
\begin{array}{l}
	x + 2y + 2z + t = 2\\
	2x + y + z + 2t = 1\\
	2x + 2y + z + t = -1\\
	x + y - 2z + 2t = -8\\
\end{array}
\right.
$$
je�li wiadomo, �e wyznacznik macierzy uk�adu wynosi $W = 12$.

}
\hspace{1.2cm}
\parbox{9cm}{
	\noindent
\hspace{4cm}
\textbf{I}

\noindent
\textbf{1}  Wyznacz pole obszaru ograniczonego liniami $y =3x - x^2$, $y = 3-x$.

\noindent
\textbf{2} Rozwi�� uk�ad r�wna� stosuj�c metod� Gaussa eliminacji:
$$
\left\{
\begin{array}{l}
	x+2 y+3 z = -5\\
	3 x+2 y+z = 5\\
	2 x+3 y+z = 4\\
\end{array}
\right .
$$

\noindent
\textbf{3} W oparciu o wzory Cramera wyznacz niewiadom� $z$ spe�niaj�c� uk�ad r�wna�:
$$
\left \{
\begin{array}{l}
	x + 2y + 2z + t = 2\\
	2x + y + z + 2t = 1\\
	2x + 2y + z + t = -1\\
	x + y - 2z + 2t = -8\\
\end{array}
\right.
$$
je�li wiadomo, �e wyznacznik macierzy uk�adu wynosi $W = 12$.
	
}\\

\vspace{0.5cm}


\noindent
\parbox{9cm}{
	\noindent
\hspace{4cm}
\textbf{I}

\noindent
\textbf{1}  Wyznacz pole obszaru ograniczonego liniami $y =3x - x^2$, $y = 3-x$.

\noindent
\textbf{2} Rozwi�� uk�ad r�wna� stosuj�c metod� Gaussa eliminacji:
$$
\left\{
\begin{array}{l}
	x+2 y+3 z = -5\\
	3 x+2 y+z = 5\\
	2 x+3 y+z = 4\\
\end{array}
\right .
$$

\noindent
\textbf{3} W oparciu o wzory Cramera wyznacz niewiadom� $z$ spe�niaj�c� uk�ad r�wna�:
$$
\left \{
\begin{array}{l}
	x + 2y + 2z + t = 2\\
	2x + y + z + 2t = 1\\
	2x + 2y + z + t = -1\\
	x + y - 2z + 2t = -8\\
\end{array}
\right.
$$
je�li wiadomo, �e wyznacznik macierzy uk�adu wynosi $W = 12$.

}
\hspace{1.2cm}
\parbox{9cm}{
	\noindent
\hspace{4cm}
\textbf{I}

\noindent
\textbf{1}  Wyznacz pole obszaru ograniczonego liniami $y =3x - x^2$, $y = 3-x$.

\noindent
\textbf{2} Rozwi�� uk�ad r�wna� stosuj�c metod� Gaussa eliminacji:
$$
\left\{
\begin{array}{l}
	x+2 y+3 z = -5\\
	3 x+2 y+z = 5\\
	2 x+3 y+z = 4\\
\end{array}
\right .
$$

\noindent
\textbf{3} W oparciu o wzory Cramera wyznacz niewiadom� $z$ spe�niaj�c� uk�ad r�wna�:
$$
\left \{
\begin{array}{l}
	x + 2y + 2z + t = 2\\
	2x + y + z + 2t = 1\\
	2x + 2y + z + t = -1\\
	x + y - 2z + 2t = -8\\
\end{array}
\right.
$$
je�li wiadomo, �e wyznacznik macierzy uk�adu wynosi $W = 12$.
	
}\\

\vspace{0.5cm}


\noindent
\parbox{9cm}{
	\noindent
\hspace{4cm}
\textbf{I}

\noindent
\textbf{1}  Wyznacz pole obszaru ograniczonego liniami $y =3x - x^2$, $y = 3-x$.

\noindent
\textbf{2} Rozwi�� uk�ad r�wna� stosuj�c metod� Gaussa eliminacji:
$$
\left\{
\begin{array}{l}
	x+2 y+3 z = -5\\
	3 x+2 y+z = 5\\
	2 x+3 y+z = 4\\
\end{array}
\right .
$$

\noindent
\textbf{3} W oparciu o wzory Cramera wyznacz niewiadom� $z$ spe�niaj�c� uk�ad r�wna�:
$$
\left \{
\begin{array}{l}
	x + 2y + 2z + t = 2\\
	2x + y + z + 2t = 1\\
	2x + 2y + z + t = -1\\
	x + y - 2z + 2t = -8\\
\end{array}
\right.
$$
je�li wiadomo, �e wyznacznik macierzy uk�adu wynosi $W = 12$.

}
\hspace{1.2cm}
\parbox{9cm}{
	\noindent
\hspace{4cm}
\textbf{I}

\noindent
\textbf{1}  Wyznacz pole obszaru ograniczonego liniami $y =3x - x^2$, $y = 3-x$.

\noindent
\textbf{2} Rozwi�� uk�ad r�wna� stosuj�c metod� Gaussa eliminacji:
$$
\left\{
\begin{array}{l}
	x+2 y+3 z = -5\\
	3 x+2 y+z = 5\\
	2 x+3 y+z = 4\\
\end{array}
\right .
$$

\noindent
\textbf{3} W oparciu o wzory Cramera wyznacz niewiadom� $z$ spe�niaj�c� uk�ad r�wna�:
$$
\left \{
\begin{array}{l}
	x + 2y + 2z + t = 2\\
	2x + y + z + 2t = 1\\
	2x + 2y + z + t = -1\\
	x + y - 2z + 2t = -8\\
\end{array}
\right.
$$
je�li wiadomo, �e wyznacznik macierzy uk�adu wynosi $W = 12$.
	
}\\

\vspace{0.5cm}


\noindent
\parbox{9cm}{
	\noindent
\hspace{4cm}
\textbf{I}

\noindent
\textbf{1}  Wyznacz pole obszaru ograniczonego liniami $y =3x - x^2$, $y = 3-x$.

\noindent
\textbf{2} Rozwi�� uk�ad r�wna� stosuj�c metod� Gaussa eliminacji:
$$
\left\{
\begin{array}{l}
	x+2 y+3 z = -5\\
	3 x+2 y+z = 5\\
	2 x+3 y+z = 4\\
\end{array}
\right .
$$

\noindent
\textbf{3} W oparciu o wzory Cramera wyznacz niewiadom� $z$ spe�niaj�c� uk�ad r�wna�:
$$
\left \{
\begin{array}{l}
	x + 2y + 2z + t = 2\\
	2x + y + z + 2t = 1\\
	2x + 2y + z + t = -1\\
	x + y - 2z + 2t = -8\\
\end{array}
\right.
$$
je�li wiadomo, �e wyznacznik macierzy uk�adu wynosi $W = 12$.

}
\hspace{1.2cm}
\parbox{9cm}{
	\noindent
\hspace{4cm}
\textbf{I}

\noindent
\textbf{1}  Wyznacz pole obszaru ograniczonego liniami $y =3x - x^2$, $y = 3-x$.

\noindent
\textbf{2} Rozwi�� uk�ad r�wna� stosuj�c metod� Gaussa eliminacji:
$$
\left\{
\begin{array}{l}
	x+2 y+3 z = -5\\
	3 x+2 y+z = 5\\
	2 x+3 y+z = 4\\
\end{array}
\right .
$$

\noindent
\textbf{3} W oparciu o wzory Cramera wyznacz niewiadom� $z$ spe�niaj�c� uk�ad r�wna�:
$$
\left \{
\begin{array}{l}
	x + 2y + 2z + t = 2\\
	2x + y + z + 2t = 1\\
	2x + 2y + z + t = -1\\
	x + y - 2z + 2t = -8\\
\end{array}
\right.
$$
je�li wiadomo, �e wyznacznik macierzy uk�adu wynosi $W = 12$.
	
}\\

\vspace{0.5cm}


\noindent
\parbox{9cm}{
	\noindent
\hspace{4cm}
\textbf{I}

\noindent
\textbf{1}  Wyznacz pole obszaru ograniczonego liniami $y =3x - x^2$, $y = 3-x$.

\noindent
\textbf{2} Rozwi�� uk�ad r�wna� stosuj�c metod� Gaussa eliminacji:
$$
\left\{
\begin{array}{l}
	x+2 y+3 z = -5\\
	3 x+2 y+z = 5\\
	2 x+3 y+z = 4\\
\end{array}
\right .
$$

\noindent
\textbf{3} W oparciu o wzory Cramera wyznacz niewiadom� $z$ spe�niaj�c� uk�ad r�wna�:
$$
\left \{
\begin{array}{l}
	x + 2y + 2z + t = 2\\
	2x + y + z + 2t = 1\\
	2x + 2y + z + t = -1\\
	x + y - 2z + 2t = -8\\
\end{array}
\right.
$$
je�li wiadomo, �e wyznacznik macierzy uk�adu wynosi $W = 12$.

}
\hspace{1.2cm}
\parbox{9cm}{
	\noindent
\hspace{4cm}
\textbf{I}

\noindent
\textbf{1}  Wyznacz pole obszaru ograniczonego liniami $y =3x - x^2$, $y = 3-x$.

\noindent
\textbf{2} Rozwi�� uk�ad r�wna� stosuj�c metod� Gaussa eliminacji:
$$
\left\{
\begin{array}{l}
	x+2 y+3 z = -5\\
	3 x+2 y+z = 5\\
	2 x+3 y+z = 4\\
\end{array}
\right .
$$

\noindent
\textbf{3} W oparciu o wzory Cramera wyznacz niewiadom� $z$ spe�niaj�c� uk�ad r�wna�:
$$
\left \{
\begin{array}{l}
	x + 2y + 2z + t = 2\\
	2x + y + z + 2t = 1\\
	2x + 2y + z + t = -1\\
	x + y - 2z + 2t = -8\\
\end{array}
\right.
$$
je�li wiadomo, �e wyznacznik macierzy uk�adu wynosi $W = 12$.
	
}\\






\end{document}