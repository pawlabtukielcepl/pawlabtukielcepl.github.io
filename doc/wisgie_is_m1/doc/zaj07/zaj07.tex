\documentclass[11pt,a4paper]{report}

\usepackage{float} %dzieki temu pakietowi rysunki pojawiaja sie tam gdzie chce po uzyciu opcji H w srodowisku figure
\usepackage[font=small,labelfont=bf]{caption}
\usepackage{natbib}
\usepackage{import}
\usepackage[tight]{subfigure}
\usepackage{tikz}
\usepackage{pgfplots}
\usepackage{polski}
\usepackage[cp1250]{inputenc}
\usepackage{indentfirst}
\usepackage{color}
\usepackage{graphics}
\usepackage{geometry}
\usepackage{amsmath,amssymb,amsfonts}
% ---- PAGE LAYOUT ----
\geometry{a4paper, top=1.5cm, bottom=1.5cm, left=1.5cm,
right=1.5cm, nohead}

\begin{document}

\begin{center}
\large{\textbf{ZESTAW ZADA� VII}}
\end{center}


\noindent
\textbf{Zadanie 1} Za pomoc� w�asno�ci ca�ki oznaczonej podanych na wyk�adzie oszacuj warto�� ca�ki $\int\limits_{0}^{\frac{\pi}{4}} \frac{4dx}{x^2+1}$:\\
\textbf{(a)} bez podzia�u przedzia�u ca�kowania,\\
\textbf{(b)} dziel�c przedzia� ca�kowania na dwie cz�ci r�wnej d�ugo�ci,\\
\textbf{(c)} dziel�c przedzia� ca�kowania na cztery r�wne cz�ci.\\

\noindent
\textbf{Zadanie 2} Oblicz ca�ki nieoznaczone:\\
\textbf{(a)} $\int (4x^2 - 3x + 5)dx$,  
\textbf{(b)} $\int\left(\frac{1}{x^2} - \frac{2}{x^3} + \frac{3}{x^4} - \frac{5}{x^5}\right)dx$,
\textbf{(c)} $\int\left(3\sqrt[3]{x} - \frac{2}{\sqrt[3]{x}}\right)dx$,\\
\textbf{(d)} $\int\left(2\sin x + 3\cos x \right)dx$, 
\textbf{(e)} $\int\left(\frac{2}{\cos^2 x} - \frac{5}{\sin^2 x}\right)dx$, 
\textbf{(f)} $\int\left(\frac{3}{\sqrt{1-x^2}} - \frac{4}{x^2 + 1}\right)dx$,\\
\textbf{(g)} $\int\left(3e^x - \frac{5}{x}\right)dx$, 
\textbf{(h)} $\int\cos(3x)dx$, 
\textbf{(i)} $\int e^{2x}\cos(3x)dx$.\\

\noindent
\textbf{Zadanie 3} Oblicz ca�ki nieoznaczone stosuj�c podane podstawienia:\\ 
\textbf{(a)} $\int e^{2x} dx$, $u = 2x$, \textbf{(b)} $\int \frac{e^{x} dx}{(3e^x - 2)^5}$, $u = 3e^x - 2$, \textbf{(c)} $\int x^3 (x^4 + 1)^{99} dx$, $u = x^4 + 1$,\\ 
\textbf{(d)} $\int \frac{\cos x dx}{\sin^3 x}$, $u = \sin x$, \textbf{(e)} $\int x^2 \sqrt{x + 1} dx$, $u = x + 1$.\\



\end{document}