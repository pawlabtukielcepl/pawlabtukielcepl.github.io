\documentclass{beamer}

\usetheme{Berlin}

\usepackage{polski}
\usepackage[cp1250]{inputenc}


\title{Egzamin z Matematyki 1 (WISGiE/I�, termin pierwszy)}
\date{05/02/2021}

\begin{document}
	
\frame{\titlepage}

\begin{frame}[t]
\frametitle{Zadanie 1 (0 - 10 pkt.)}

\noindent
\Large{
Oblicz pochodne: 
$\left(\frac{3}{x^5} - \frac{5}{\sqrt[5]{x^9}}\right)'$, 
$\left( \frac{\arcsin x}{5^x} \right)'$, 
$\left( \sin x \cos(x^5) \right)'$.\\
}

\end{frame}

\begin{frame}[t]
\frametitle{Zadanie 2 (0 - 10 pkt.)}

\noindent
\Large{
Zapisz wz�r Taylora dla funkcji $f(x) = \sqrt{x}$ w okolicy $x_0 = 4$ z dok�adno�ci� do wyraz�w drugiego rz�du. Wykorzystaj uzyskany wz�r do wyznaczenia przybli�onej warto�ci $\sqrt{4,2}$.
}

\end{frame}

\begin{frame}[t]
\frametitle{Zadanie 3 (0 - 10 pkt.)}

\noindent
\Large{
Wyznacz przedzia�y monotoniczno�ci i ekstrema lokalne funkcji: 
$y = 3 x^4-8 x^3-18 x^2+72 x$.
}
\end{frame}

\begin{frame}[t]
\frametitle{Zadanie 4 (0 - 10 pkt.)}

\noindent
\Large{
Oblicz ca�k�: 
$\int \frac{3x + 2}{x^2 + 4x + 4} dx$.
}

\end{frame}

\begin{frame}[t]
\frametitle{Zadanie 5 (0 - 10 pkt.)}

\noindent
\Large{
 Oblicz ca�ki oznaczone: 
$\int\limits_{1}^{4} \left(\sqrt{x} - \frac{1}{x^2}\right) dx$, 
$\int\limits_{0}^{2} \frac{x dx}{\sqrt{3x^2 + 4}}$.
}


\end{frame}

\begin{frame}[t]
\frametitle{Zadanie 6 (0 - 10 pkt.)}

\noindent
\Large{
Wyznacz pole obszaru ograniczonego liniami  $y = x^2 - 3x$, $y = -x^2 -2x + 1$. Wykonaj rysunek!
}

\end{frame}

\begin{frame}[t]
\frametitle{Zadanie 7 (0 - 20 pkt.)}\


\noindent
\Large{
W oparciu o definicj� oblicz pochodn� podanej funkcji $f(x) = 2x^2-3x+1$ w punkcie $x_0 = 1$. Zapisz r�wnanie stycznej do wykresu funkcji w punkcie $(x_0,f(x_0))$, naszkicuj pogl�dowy wykres funkcji oraz stycznej.
}

\end{frame}

\begin{frame}[t]
\frametitle{Zadanie 8 (0 - 20 pkt.)}

\noindent
\Large{
W oparciu o rachunek ca�kowy wyznacz po�o�enie �rodka ci�ko�ci obszaru ograniczonego liniami $y = x^2$, $y = 2x$, je�li wiadomo, �e pole obszaru wynosi $S = \frac 43$.
}
\end{frame}


\end{document}