\documentclass{beamer}

\usetheme{Berlin}

\usepackage{polski}
\usepackage[cp1250]{inputenc}


\title{Egzamin z Matematyki 1 (WISGiE/I�/N, sesja poprawkowa)}
\date{12/02/2023}

\begin{document}
	
\frame{\titlepage}

\begin{frame}[t]
\frametitle{Zadanie 1 (0 - 10 pkt.)}

\noindent
\Large{
Oblicz pochodne: 
$$\left(\frac{1}{x^3} - \frac{1}{\sqrt{x}}\right)'$$ 
$$\left( \frac{\text{arctg}\,x}{\sin x} \right)'$$ 
$$\left( x^2e^{\cos x} \right)'$$
}

\end{frame}

\begin{frame}[t]
\frametitle{Zadanie 2 (0 - 10 pkt.)}

\noindent
\Large{
Zapisz wz�r Taylora dla funkcji 
$$f(x) = \sqrt[3]{x}$$ 
w okolicy $x_0 = 8$ z dok�adno�ci� do wyraz�w drugiego rz�du. Oblicz za pomoc� tego wzoru przybli�on� warto�� $\sqrt[3]{7{,}9}$.
}

\end{frame}

\begin{frame}[t]
\frametitle{Zadanie 3 (0 - 10 pkt.)}

\noindent
\Large{
Wyznacz przedzia�y monotoniczno�ci i ekstrema lokalne funkcji: 
$$y = 6 x^4 + 8 x^3 - 3 x^2 - 6 x$$
}
\end{frame}

\begin{frame}[t]
\frametitle{Zadanie 4 (0 - 10 pkt.)}

\noindent
\Large{
Oblicz ca�k�: 
$$\int \frac{3 x-4}{x^2-4 x+4} dx$$
}

\end{frame}

\begin{frame}[t]
\frametitle{Zadanie 5 (0 - 10 pkt.)}

\noindent
\Large{
Oblicz ca�ki oznaczone: 
$$\int\limits_{1}^{4} \left(\sqrt{x} - \frac{1}{x^2}\right) dx$$ 
$$\int\limits_{1}^{2} \frac{x dx}{\sqrt{3x^2 + 13}}$$
}


\end{frame}

\begin{frame}[t]
\frametitle{Zadanie 6 (0 - 10 pkt.)}

\noindent
\Large{
Wyznacz pole obszaru ograniczonego liniami  
$$y = 4 - x^2,\ \ y = 3$$ 
Wykonaj rysunek!
}

\end{frame}

\begin{frame}[t]
\frametitle{Zadanie 7 (0 - 10 pkt.)}\


\noindent
\Large{
W oparciu o definicj� oblicz pochodn� podanej funkcji 
$$f(x) = 4x^2-7x-2$$ 
w punkcie $x_0 = 1$. Zapisz r�wnanie stycznej do wykresu funkcji w punkcie $(x_0,f(x_0))$, naszkicuj pogl�dowy wykres funkcji oraz stycznej.
}

\end{frame}

\begin{frame}[t]
\frametitle{Zadanie 8 (0 - 10 pkt.)}

\noindent
\Large{
Dane s� wektory 
$$\vec{u} = 2\vec{j} - 3\vec{i} + 6\vec{k} \text{ oraz } \vec{v} = -6\vec{k} + 3\vec{i} + 6\vec{j}$$ Wyznacz kosinus k�ta pomi�dzy nimi oraz pole tr�jk�ta rozpi�tego na tych wektorach.
}
\end{frame}


\end{document}