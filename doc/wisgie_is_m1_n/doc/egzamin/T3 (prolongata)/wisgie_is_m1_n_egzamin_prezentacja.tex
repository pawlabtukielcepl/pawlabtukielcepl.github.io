\documentclass{beamer}

\usetheme{Berlin}

\usepackage{polski}
\usepackage[cp1250]{inputenc}


\title{Egzamin z Matematyki 1 (WISGiE/I�/N, przed�u�ona sesja poprawkowa)}
\date{19/02/2023}

\begin{document}
	
\frame{\titlepage}

\begin{frame}[t]
\frametitle{Zadanie 1 (0 - 10 pkt.)}

\noindent
\Large{
\textbf{Zadanie 1 (0-10 pkt.)} Oblicz pochodne: 
$$\left( \frac{1}{x^2} + \sqrt{x} \right)'$$ 
$$\left( \frac{\arcsin x}{\ln x} \right)'$$ 
$$\left( x^4 \sin(x^2) \right)'$$
}

\end{frame}

\begin{frame}[t]
\frametitle{Zadanie 2 (0 - 10 pkt.)}

\noindent
\Large{
Zapisz wz�r Taylora dla funkcji 
$$f(x) = \sqrt{x}$$ 
w okolicy $x_0 = 1$ z dok�adno�ci� do wyraz�w drugiego rz�du.\\ 
Oblicz za pomoc� tego wzoru przybli�on� warto�� 
$$\sqrt{1{,}1}$$
}

\end{frame}

\begin{frame}[t]
\frametitle{Zadanie 3 (0 - 10 pkt.)}

\noindent
\Large{
Wyznacz przedzia�y monotoniczno�ci i ekstrema lokalne funkcji: 
$$y = 6 x^4 + 8 x^3 - 3 x^2 - 6 x$$
}
\end{frame}

\begin{frame}[t]
\frametitle{Zadanie 4 (0 - 10 pkt.)}

\noindent
\Large{
Oblicz ca�k�: 
$$\int \frac{3 x + 1}{x^2 - 4 x + 3} dx$$
}

\end{frame}

\begin{frame}[t]
\frametitle{Zadanie 5 (0 - 10 pkt.)}

\noindent
\Large{
Oblicz ca�ki oznaczone: 
$$\int\limits_{1}^{4} \left(3x^2 +  \frac{2}{x^2}\right) dx$$ 
$$\int\limits_{-1}^{1} x^2 ( x^3 + 1)^ 3 dx$$
}


\end{frame}

\begin{frame}[t]
\frametitle{Zadanie 6 (0 - 10 pkt.)}

\noindent
\Large{
Wyznacz pole obszaru ograniczonego liniami  
$$y = 4 - x^2,\ \ y = 3$$ 
Wykonaj rysunek!
}

\end{frame}


\end{document}