\documentclass[11pt,a4paper]{report}

\usepackage{float} %dzieki temu pakietowi rysunki pojawiaja sie tam gdzie chce po uzyciu opcji H w srodowisku figure
\usepackage[font=small,labelfont=bf]{caption}
\usepackage{natbib}
\usepackage{import}
\usepackage[tight]{subfigure}
\usepackage{tikz}
\usepackage{pgfplots}
\usepackage{polski}
\usepackage[cp1250]{inputenc}
\usepackage{indentfirst}
\usepackage{color}
\usepackage{graphics}
\usepackage{geometry}
\usepackage{amsmath,amsfonts,amssymb}
% ---- PAGE LAYOUT ----
\geometry{a4paper, top=1.5cm, bottom=1.5cm, left=1.5cm,
right=1.5cm, nohead}
\begin{document}
	
\begin{center}
\large{\textbf{Egzamin z matematyki 1 (WI�GiE/I�/N, przed�u�ona sesja poprawkowa), 19/02/2023}}
\end{center}

\noindent
\textbf{Zadanie 1 (0-10 pkt.)} Oblicz pochodne: 
$\left( \frac{1}{x^2} + \sqrt{x} \right)'$, 
$\left( \frac{\arcsin x}{\ln x} \right)'$, 
$\left( x^4 \sin(x^2) \right)'$.\\


\noindent
\textbf{Zadanie 2 (0-10 pkt.)} Zapisz wz�r Taylora dla funkcji $y = \sqrt{x}$ w okolicy $x_0 = 1$ z dok�adno�ci� do wyraz�w drugiego rz�du. Oblicz za pomoc� tego wzoru przybli�on� warto�� $\sqrt{1{,}1}$.\\


\noindent
\textbf{Zadanie 3 (0-10 pkt.)}  Wyznacz przedzia�y monotoniczno�ci i ekstrema lokalne funkcji: 
$y = 6 x^4 + 8 x^3 - 3 x^2 - 6 x$.\\

\noindent
\textbf{Zadanie 4 (0-10 pkt.)}  Oblicz ca�k�: 
$\int \frac{3 x + 1}{x^2 - 4 x + 3} dx$.\\

\noindent
\textbf{Zadanie 5 (0-10 pkt.)}  Oblicz ca�ki oznaczone: 
$\int\limits_{1}^{4} \left(3x^2 +  \frac{2}{x^2}\right) dx$, 
$\int\limits_{-1}^{1} x^2 ( x^3 + 1)^ 3 dx$.\\

\noindent
\textbf{Zadanie 6 (0-10 pkt.)}  Wyznacz pole obszaru ograniczonego liniami  $y = 4 - x^2$, $y = 3$. Wykonaj rysunek!\\ 





\end{document}