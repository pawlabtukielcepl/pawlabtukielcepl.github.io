\documentclass[11pt,a4paper]{report}

\usepackage{float} %dzieki temu pakietowi rysunki pojawiaja sie tam gdzie chce po uzyciu opcji H w srodowisku figure
\usepackage[font=small,labelfont=bf]{caption}
\usepackage{natbib}
\usepackage{import}
\usepackage[tight]{subfigure}
\usepackage{tikz}
\usepackage{pgfplots}
\usepackage{polski}
\usepackage[cp1250]{inputenc}
\usepackage{indentfirst}
\usepackage{color}
\usepackage{graphics}
\usepackage{geometry}
\usepackage{amsmath,amssymb,amsfonts}
% ---- PAGE LAYOUT ----
\geometry{a4paper, top=1.5cm, bottom=1.5cm, left=1.5cm,
right=1.5cm, nohead}

\begin{document}

\begin{center}
\large{\textbf{ZESTAW ZADA� III}}
\end{center}

\noindent
\textbf{Zadanie 1}\\
\textbf{(a)} Zapisz wz�r Taylora dla funkcji $f(x) = \frac{2x}{2-x}$ z dok�adno�ci� do dw�ch wyraz�w w okolicy $x_0 = 1$; wykorzystaj otrzymany wz�r do przybli�enia warto�ci funkcji dla $x = 0{,}9$,\\
\textbf{(b)} Zapisz wz�r Taylora dla funkcji $f(x) = \cos x$ z dok�adno�ci� do wyraz�w $2$--go rz�du w okolicy $x_0 = 0$. Za pomoc� uzyskanego wzoru oblicz przybli�on� warto�� $\cos 15^{\circ}$ (warto�� podana przez kalkulator: $\cos 15^{\circ} \approx 0{,}96596$; w obliczeniach przyjmij: $\pi\approx 3{,}14$).\\


\noindent
\textbf{Zadanie 2} W oparciu o regu�� de l'Hospitala oblicz poni�sze granice:\\
\textbf{(a)} $\lim\limits_{x\rightarrow 2} \frac{x^3+5 x^2-2 x-24}{x^3-2 x^2-3 x+6}$, 
\textbf{(b)} $\lim\limits_{x\rightarrow 0} \frac{x\sin x}{e^{-2x} - 1 + 2x}$, 
\textbf{(c)} $\lim\limits_{x\rightarrow 0}\frac{\sin(x^2)}{\ln(\cos x)}$, 
\textbf{(d)} $\lim\limits_{x\rightarrow\infty}\frac{e^x}{x^2}$.\\

\noindent
\textbf{Zadanie 3} \\ 
Wyznacz przedzia�y monotoniczno�ci i ekstrema lokalne podanych funkcji:\\
\textbf{(a)} $y = -2 x^3+4 x^2+8 x+10$, \textbf{(b)} $y = -x^4+x^3+6 x^2-9 x + 5$, \textbf{(c)} $y = 3x + \frac{1}{x^3}$,\\
\textbf{(d)} $y = x^5 + (1-x)^5$,  \textbf{(e)} $y = x^4(2x-3)^6$, \textbf{(f)} $y = \frac{2x^2-5x+2}{3x^2-10x+3}$, \textbf{(g)} $y = x^2\ln x$.\\



\end{document}