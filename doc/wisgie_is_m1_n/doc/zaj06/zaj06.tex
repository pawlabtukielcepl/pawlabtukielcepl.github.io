\documentclass[11pt,a4paper]{report}

\usepackage{float} %dzieki temu pakietowi rysunki pojawiaja sie tam gdzie chce po uzyciu opcji H w srodowisku figure
\usepackage[font=small,labelfont=bf]{caption}
\usepackage{natbib}
\usepackage{import}
\usepackage[tight]{subfigure}
\usepackage{tikz}
\usepackage{pgfplots}
\usepackage{polski}
\usepackage[cp1250]{inputenc}
\usepackage{indentfirst}
\usepackage{color}
\usepackage{graphics}
\usepackage{geometry}
\usepackage{amsmath,amssymb,amsfonts}
% ---- PAGE LAYOUT ----
\geometry{a4paper, top=1.5cm, bottom=1.5cm, left=1.5cm,
right=1.5cm, nohead}

\begin{document}

\begin{center}
\large{\textbf{ZESTAW ZADA� VI}}
\end{center}

\noindent
\textbf{Zadanie 1} Wyznacz ca�ki nieoznaczone stosuj�c wz�r na ca�kowanie przez cz�ci:\\
\textbf{(a)} $\int x\cos(2x) dx$, \textbf{(b)} $\int x^2 e^{-2x}dx$, 
\textbf{(c)} $\int \frac{\ln x dx}{\sqrt{x}}$, \textbf{(d)} $\int\ln^2 x dx$, \textbf{(e)} $\int \arcsin x dx$, \textbf{(f)} $\int x^2 \,\text{arctg}\,x dx$,\\ 
\textbf{(g)} $\int\sin^2 x dx$, \textbf{(h)} $\int e^{2x}\cos(3x)dx$.\\

\noindent
\textbf{Zadanie 2} Wyznacz ca�ki nieoznaczone:\\
\textbf{(a)} $\int \frac{5 x dx}{6 x^2+13 x+6}$, \textbf{(b)} $\int\frac{(10 x-14)dx}{4 x^2-12 x+9}$, \textbf{(c)} $\int\frac{(3 x-4)dx}{x^2-6 x+13}$, 
\textbf{(d)} $\int\frac{-2 x^3-10 x^2-x+3}{x^4-x^2} dx$, \textbf{(e)} $\int \frac{-x^3+x^2+2 x+2}{x^4+3 x^3+3 x^2+3 x+2} dx$.\\



\end{document}