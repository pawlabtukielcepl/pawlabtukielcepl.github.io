\documentclass[11pt,a4paper]{report}

\usepackage{float} %dzieki temu pakietowi rysunki pojawiaja sie tam gdzie chce po uzyciu opcji H w srodowisku figure
\usepackage[font=small,labelfont=bf]{caption}
\usepackage{natbib}
\usepackage{import}
\usepackage[tight]{subfigure}
\usepackage{tikz}
\usepackage{pgfplots}
\usepackage{polski}
\usepackage[cp1250]{inputenc}
\usepackage{indentfirst}
\usepackage{color}
\usepackage{graphics}
\usepackage{geometry}
\usepackage{amsmath,amssymb,amsfonts}
% ---- PAGE LAYOUT ----
\geometry{a4paper, top=1.5cm, bottom=1.5cm, left=1.5cm,
right=1.5cm, nohead}

\begin{document}

\begin{center}
\large{\textbf{ZESTAW ZADA� I}}
\end{center}

\noindent
\textbf{Zadanie 1} Rozwi�� nier�wno�ci:\\
\textbf{(a)} $x^2-2x\geq 0$, \textbf{(b)} $9-4x^2\leq 0$, \textbf{(c)} $x^2 + 1 > 0$, \textbf{(d)} $-2x^2-x+3\geq 0$,\\ 
\textbf{(e)} $x^4+3 x^3+2 x^2 > 0$, \textbf{(f)} $x^3+2 x^2-x-2 \geq 0$, \textbf{(g)} $x^4-5 x^2+4 > 0$.\\ 

\noindent
\textbf{Zadanie 2} \\
\textbf{(a)} Dla jakich $x$ wyra�enie $\frac{2}{x-1} - \frac{3}{2x} + \frac{4}{x^2-x}$ ma sens? Zapisz je w postaci jednego u�amka nieskracalnego.\\
\textbf{(b)} Skr�� u�amki $\frac{x^2-x}{x^3-3x^2}$, $\frac{x^2-4}{x^2+4x+4}$, $\frac{x^2-2 x-3}{x^2-x-6}$.\\
\textbf{(c)} Wyznacz wskazan� niewiadom� z r�wna�:  $\frac{ab}{c} = d$, $a = ?$; $ab + c = bd + e$, $b = ?$; $\frac{a}{3c+2} = b$, $c = ?$.\\
 
\noindent
\textbf{Zadanie 3} Rozwi�� nier�wno�ci:\\
\textbf{(a)} $\frac{x-1}{x^2-2x}\geq 0$, \textbf{(b)} $\frac{2}{2x+3}\leq 2$, \textbf{(c)} $\frac{2x}{x+1}\leq \frac{3x+2}{x+4}$.\\

\noindent
\textbf{Zadanie 4}\\
\textbf{(a)} stosuj�c definicj� funkcji trygonometrycznych dowolnego k�ta oblicz (o ile to mo�liwe) warto�ci funkcji sinus, kosinus, tangens i kotangens dla k�t�w: $0^{\circ}$, $90^{\circ}$, $180^{\circ}$, $270^{\circ}$, $135^{\circ}$ i $225^{\circ}$,\\
\textbf{(b)} stosuj�c wzory redukcyjne oblicz warto�ci funkcji trygonometrycznych sinus i kosinus dla k�t�w: $120^{\circ}$, $240^{\circ}$, $\frac{5\pi}{4}$, $\frac{5\pi}{3}$.\\

\noindent
\textbf{Zadanie 5}\\
\textbf{(a)} Zapisz wyra�enia w postaci $2^{\alpha}$, gdzie $\alpha$ -- pewna liczba wymierna: $\sqrt{2\sqrt[3]{4}}$, $\frac{\sqrt[3]{2}\sqrt[4]{8}}{(\sqrt[5]{16})^3}$.\\
\textbf{(b)} Rozwi�� r�wnania $4^{x^2} - 8^x = 0$, $3^{\frac{1}{x}} = 3^{2x}$.\\
\textbf{(c)} Rozwi�� nier�wno�ci $\left(\frac{1}{2}\right)^{x^2} - \frac{1}{16} \leq 0$, $2^{2x} - 5\cdot 2^x \leq 4$.\\

\noindent
\textbf{Zadanie 6}\\
\textbf{(a)} Oblicz: $\log_2 \sqrt[3]{32}$, $\log \sqrt{0{,}0001}$, $\ln \frac{1}{\sqrt[5]{e^3}}$, $\log_{\sqrt{2}} \frac{1}{16}$\\
\textbf{(b)} Oblicz $\log 0{,}2 - \log 2$, $\log_2 \sqrt[3]{6} - \frac 13\log_2 3$, $16^{1-\log_4 3} + 2\cdot 5^{-\log_5 9}$, $\log_9 5\cdot \log_{25}27$\\
\textbf{(c)} Rozwi�� r�wnania i nier�wno�ci: $\log_2\left(\log_3\left(\log_4 x\right)\right) = 0$, $\log_{\frac 12} (2x - 5) < -4$, $\log_{\frac 13} \left(\log_4 (x^2 - 5)\right) > 0$\\



\end{document}